%For those not familiar with LaTeX, you can just put contributions in comments (preceded with % signs)
%so that LaTeX users can update the whole document
%LaTeX is preferred because, despite not being WYSIWYG, it is _very_ amenable to revision control, unlike
%MSOffice or LibreOffice software
%
%Try to keep one sentence per line for ease of maintenance
\documentclass{article}
\usepackage{fullpage}
\usepackage{graphicx}
\usepackage{geometry}
\usepackage{xcolor,soul}

\sethlcolor{yellow}

\graphicspath{./images}

\newcommand{\specificCite}[1]{\tiny #1 \normalsize}

\title{Barbaria: Creating a Consistent Narrative for lands beyond the Roman border}

\begin{document}
	\newgeometry{left=1.5cm,bottom=1.5cm,right=1.5cm,top=1.5cm}
	
	\maketitle
	
	\section{Introduction}
	\label{sec:intro}
	
	The origin of the Balts/Slavs/Goths are a contentious issue, and the CK2 culture set up has some noticeable issues, not the least of which is very early Slavic migration to Novgord, which seems to clash with a later settlement of Northern Russia by Slavs in literature \cite{EmergenceOfRussia}.
	
	Before getting into the meat of the issue, a few notes:
	
	\begin{itemize}
		\item It is doubtful that many archaeological complexes were tied to a single culture (in CK3 terms) and it is complicated by cultures borrowing materially from each other.
		\item Actual cultures (in CK3 terms) probably spanned the borders between Archaeological Complexes.
		\item Actual ethnogenesis is a complicated process, and the hybridization/divergence system in CK3, while a huge step forward from CK2, is a gross oversimplification.
		\item We want a long-standing Baltic Veneti culture interfacing with Przeworsk, which was the model in CK2. This would also make the Neuri of Heroditus’ time Balts, which seems to be a consensus.
		\item The Urheimat of the Slavs should on the border of Baltic and Sarmatian zones, mostly to identify the “Scythian Farmers”.
		Florin Curta’s argument in favor of a Dacian-region is a minority position; although he has a point about ethnicity being a modern external construct being imposed backwards, it fails to cover well the expansion of Slavic languages, which is the central concern in this article.
		In which case the Slavic Urheimat should be closer to the generally agreed center.
		\item LT-Rascek is of the opinion that the Slavs had a later ethnogensis, and for linguistic reasons, it had to initial occur alongside the Balts.
		The map (Map 17 of \cite{HeatherEmpiresAndBarbarians}) of all the proposed Slavic Urheimat has only a few that would fit, namely G: “Baran/Godlowski – Carpathian”, C: “Rusanova – Polesia”, and “D: Tretiakov – Kievan”.
		Considering the idea that the Slavs were essentially a border people taking influences from Przeworsk and Chernyakhov, that narrows the region down to the northern Chernyakhov complex, with the Baltic Veneti occupying Polesia
		\begin{itemize}
			\item LT-Rascek thinks the references to the Dnieper Balts strengthens the later migration argument.
			The Kolochin \& Moshchiny cultures were probably transitional as Slavicization occurred and the timeline matches pretty well. So they should either be divergences from Slavic or hybrid Slavic/Baltic cultures.
		\end{itemize}
		\item Heather's "elite migration" arguments are probably a good model for what happened with the Goths and Vandals and explain the decline in material goods in the Wielbark and Przeworsk zones, especially considering the best raiding target (Rome) was historically to the south.
		Much like how nomadic elites were built on pillage and trade with settled groups, if we think that Oium represented a Gothic dominated pillage/trade confederation, it would make sense that the Przeworsk and Wielbark zones would become denuded of rulers in time.
		With some tradition modifications, we can make these cultures model more peasant/low holding and easier to displace by the Slavs.
		\item Landless gameplay unlocks powerful, useful mechanics for both the evolution of Germanic identities during migration (e.g., Vandalic, Suavi, Gothic), as well as giving us license to revisit our CK2 assumptions on how to model cultures and distributions.
		\item The mismash of Bolghar cultures conflicts somewhat with history; the Akatziri were likely a Hunnic branch \cite{KimHuns,OttoHuns} and the Bolghar's were either a component of the Huns (per Kim) or new arrivals of a sister branch.
		Given CK2 has the Bolghars as both separate from Hunnic Culture and using Scythian religion, the author presumes the intention is the second.
		\item CK3 added functional landless gameplay, which unlocks some options for representing certain aspects of history, such as roving warbands or nomadic groups that we can more readily split the political and cultural/religious maps.
		So these maps will focus primarily on the culture/religion changes while leaving the political map for a different document.
		\item The CK2 split of Germanic cultures into North/West/Central/East Germanic has some very oddities:
		\begin{itemize}
			\item East Germanic is enormous, with 15 cultures in it, including some of dubious links to East Germanic languages (Old Bavarian, Burgundian)
			\item Central Germanic is mostly Suebic cultures as well as the later German culture, but does \textit{not} include Suebian culture itself, which is classified as \textit{West Germanic} instead!
			\item Some of the Suebic peoples, like the Auiones, are put in the North Germanic culture block despite Tacitus linking them to the Suebic Angles and Reudingians.
			\item Old Saxon is listed as a Central Germanic culture, but was also frequently linked to the Franks (and likely spoke a similar language to the Anglos if we accept North Sea Germanic)
			\item The Angles, Jutes, Auiones, and Reudingians occupy and odds space between both the Suebic and North Germanics, both in terms of locations and habits, not fitting well in either group.
		\end{itemize}
		With the increased flexibility of CK3's culture system, we can reorganize the Germanic peoples into a more rational setup to better account for these sorts of nuances, in a way CK2 could not support readily
		\item \textit{WtWSMS} should really consolidate research into something indexable, like papers or articles
	\end{itemize}
	
	This article is divided into the following sections:
	
	\begin{itemize}
		\item Section \ref{sec:high_level_overview}, a high level overview providing a more narrative approach to the problem
		\item Section \ref{sec:timeline}, a more period-by-period view of the evolution of the culture map
		\item Section \ref{sec:culture_review}, reviewing changes to individual cultures
		\item Section \ref{sec:Bibliography}, for sources for future review
		\item Section \ref{sec:appendix_new_ethnicities}, for a detailed breakdown of ethnicity math
	\end{itemize}
	
	\section{High Level Overview}
	\label{sec:high_level_overview}
	
	\subsection{General Archaeological Complex Relations}
	In this diagram, the author tries and links a number of Archaeological complexes in terms of both influence (thick lines) and succession \& contribution (thin lines).
	The only unambiguous Germanic complex is the Jastorf,  the only unambiguous Slavic complex is the Kiev complex, and the Sarmatians are unambiguously Sarmatian.
	All the rest are debated, but the author think marking the Brushed Pottery culture as Baltic and the La Tene as Celtic is defensible and traditional.
	
	\begin{figure}[h]
		\centering
		\includegraphics[width=0.75\textwidth]{./images/archeological_complex_relations.png}
		\caption{Archeological Complex Relations.\newline \small \textit{Relations between various archaeological complexes.
				The only truly unambiguous cultures are Germanic Jastorf, Baltic Brushed Pottery, Sarmatian, and Slavic Kiev Culture.
				The remainder all show some degree of cultural admixture.}}
	\end{figure}
	
	\subsection{General Narrative}
	It remains difficult to know precisely what the situation in Barbaricum was before the Middle Ages, we can at least build a general narrative for self-consistency.
	The author presume there was an initial phase of Germanization prior to the Goths arrival and that \textit{Getica} is at least partly historical; there’s good genetic evidence that the Goths (or at least the elites) were originally from Scandinavia and their migration to the Vistula precipitated a number of changes among the Przeworsk, as well as likely the Zarubintsy – possibly breaking the continuum between the two to allow the Zarubintsy to fully Slavicize.
	
	It is likely the Lusatian culture was Baltic, or at least non-Germanic.
	Most models for the spread of the Germanic languages have the mouth of the Vistula Germanizing between 500 BC and nearly the whole Vistula Germanizing by Augustus’ time \cite{HeatherEmpiresAndBarbarians}.
	This would make the Pomeranian Culture, the antecedent of Przeworsk and Oksywie Cultures, the “Proto-East Germanic” Culture, although with significant Baltic/non-Germanic influences.
	The later two were probably fully “German”, although the author suspects that the Jastorf played a bigger role than Norse in this period.
	As for the Balts, the author presumes their original extent was at least the generally agreed Baltic Hydronym region \cite{HeatherEmpiresAndBarbarians,BalticHydronyms}.
	
	The Przeworsk would be an admixture and still borrow and exchange with the Zarubintsy, which tend to be viewed as either a sister group of Przeworsk or its eastern extension.
	Additionally, the Chernoles Culture of 1025-700 BC fits between the (likely) Baltic Milograd and the Proto-Scythians, while makes them possibly the "Scythian Farmers" of Herodotus, which underwent changes with the rise of the Scythian cultures after 500 BC.
	So around 200 BC it’d probably look something like this:
	
	\begin{figure}[h!]
		\centering
		\includegraphics[width=1.00\textwidth]{./images/archeological_complexes_200BC.png}
		\caption{Archaeological Complexes ca 200 BC.
			\tiny Zarubintsy extents taken from \cite{IndoEuroEncyclopedia}, Przeworsk extents taken from \cite{PrzeworskHistory} (Early Phase), Baltic Extents (including Zarubintsy) taken from \cite{BalticHydronyms}, Saami extents taken from \cite{LaplandSaami}, and Balto-Finnic is extrapolated.
			Note that Przeworsk is abutting the La Tene Zone in modern Bohemia and Jastorff in Germania, serving as a meeting point between Balto-Slavic, Celtic, and Germanic influences.}
	\end{figure}
	
	Around the IIIrd Century BC, Pomeranian gave way to Oksywie and Przeworsk.
	The former we don’t have as much data, as it is not a culture in CK2 WtWSMS, but being right along the Baltic coast, it was the first to receive the settlers/colonizers from Scandinavia.
	For this reason and the otherwise limited expanse of Germanic during the period, the author associatse them with the sudden appearance of the generally agreed Germanic Bastarnae and Skirians at the Greco-Roman border ca. 200 BC.
	So the original Skirian appearance in the South probably portended the migrations that would follow, and the Skirians and Bastarnae were probably semi-Germanic early migrators.
	
	The author suspects that this nucleus would remain near the Vistula mouth and, besides pushing the Skirians and Bastarnae to the Carpathians, primarily be the final Germanizing of Oksywie and Przeworsk.
	That would remain the process for the next few hundred years.
	Then, the Wielbark culture arrives in the Ist Century AD.
	
	As Heather points out \cite{HeatherEmpiresAndBarbarians}, Wielbark expanded at the expense of Przeworsk, down the Vistula river and this lines up with \textit{Getica} saying the Goths arrived from Scandinavia and eventually end up in Oium (generally agreed to be modern Ukraine, at least partially).
	Under this narrative, Wielbark represents the “original” Gothic culture, or more precisely, its form prior to the Chernyakhov period where Goths dominated what is now Moldova and (at least) Ukraine west of the Dneiper.
	Over the following centuries, Wielbark would expand down the Vistula, partly breaking the continuity between Zarubnitsy and Przeworsk and likely facilitated the further Germanization of Przeworsk. 
	That said, Przeworsk was also under change during the period; by the late Roman Period, part of its material culture arose in the headwaters of the Tisza \cite{PrzeworskHistory}. This is likely related to the proto-Vandalic groups, which seem to be attested to in the region circa AD 270s.
	
	Wielbark would continue extending down the Vistula for the next few hundred years in a sort of transient way \cite{HeatherEmpiresAndBarbarians}, where it then came in contact with the Sarmatians and/or Slavs.
	There, it became a new Chernyakhov culture from the IInd to the Vth centuries.
	The author suspects Chernyakhov were truly multi-ethnic with a dominant Germanic (namely Gothic) component.
	The author further suspects there’s an element of truth to Oium in \textit{Getica}, although it is probably a little less of an empire and more a broad, confederal Gothic dominance of the early Slavs, Dacians, and Sarmatians, who intermingled in a way that creates a common shared material culture.
	This would last until the late IVth and early Vth Centuries, when the Huns attack, sending the Goths fleeing into the Roman Empire and otherwise generally shuffling around peoples and polities of eastern barbaria. 
	
	\begin{figure}[h!]
		\centering
		\includegraphics[width=1.00\textwidth]{./images/archeological_complexes_350AD.png}
		\caption{Archaeological Complexes ca. AD 350.
			\newline\tiny Wielbark has expanded down to modern Ukraine, giving birth to the Chernkyakhov Archeological Complex.
			Przeworsk has found itself somewhat marginalized, but components have moved into the chaotic lands of former Dacia; this is either Gepid or Vandalic, probably Gepid is \textit{Getica} is referencing some actual conflicts. Known Iazyges lands provided for comparison. Chernyakov extents taken from \cite{HeatherEmpiresAndBarbarians,IndoEuroEncyclopedia}, Kyiv extents taken from \cite{IndoEuroEncyclopedia} under Early Kolochin}
	\end{figure}
	
	The Germanic Groups that fled into the Empire, namely the Goths and Vandals, were probably an elite, aristocratic component and their fleeing is what precipitated the collapse of Germanic Wielbark and Prezworsk in the late IVth and early Vth centuries in Ukraine and Poland \cite{HeatherEmpiresAndBarbarians}.
	Other groups, like the Slavs, were probably subordinate or simply poorer (Gepids, Heruli, Skirians) and the Huns would make use of them extensively as soldiers because they were able to turn on their former masters/wealthier counterparts or were, like the Goths and Vandals, pushed away from the general Chernyakhov region in favor of the new Huns.
	This would mirror the collapse of the Rouran in the East and the rise of the Gokturks centuries later.
	
	The author strongly suspects the Slavs filled the vacuum in Ukraine and Southern Poland left by the Germans fleeing as this region overlaps well with the better attestation of Attila’s Hunnic realm.
	The relatively flat social structure of the Slavs during the time also probably played a role in the Slaviziation of the region, given the Germanic elite, skilled component ended up fleeing towards the empire (which Heather hypothesizes \cite{HeatherEmpiresAndBarbarians}).
	This dovetails with CK2 WtWSMS’s sudden appearance of Lechitic culture in modern Poland around 525 and they seem to be the most likely group to replace the Germans there, if we go by the Slavs living in the woods between the Baltic and Sarmatian peoples (which plays well with keep Balto-Slavic relatively adjoined until late in their history and the notion that the proto-Slavs are the "Scythian Farmers" of antiquity), before reaching their final form in the Kyiv/Kolochin groups.
	
	So, by the early Vth Century, the Huns have utterly reorganized the landscape culturally and groups have moved around significantly, as was what regularly happened in what is now the Mongolic and Turkic Steppe.
	Damage to the Przeworsk and Wielbark and the displacement of the Skirians and Heruli from Moldova and the Gepids from Slovakia opened up significant lands to Slavic settlement, entirely remaking the map.
	
	From here, the Slavs would move up the Vistula, especially when the Lombards flee, taking the last of the well-organized Germanic able to counteract Slavization, and the collapse of the Huns in the Vth Century creates room on the lower Danube for the proto-Sklaveni and proto-Antes to settle, where the Byzantines historically first met them.
	So by the early-to-middle VIth Century, Germanic cultures on the Vistula are gone, which coincides with the rise of new material cultures in the VIth and VIIth centuries that are unambiguously Slavic, especially in regions where the Eastern Germanics once lived.
	
	So around the 550s, the proto-West and proto-South Slavs are in position to spread, while the Dneiper Balts remain in time for the proto-East Slavs to push north and steadily displace them in a process that wouldn’t be completed until the early Middle Ages \cite{EmergenceOfRussia} and would more readily justify oral history recording their existence (compared to the seeming lack of oral history where the West Slavs displaced the East Germanics).
	
	The Avars arrive crica 560 and eventually bring the Slavs under their domination, helping fuel their push to the south and west that would complete the Slavization of much of the Balkans by the proto-South Slavs, push the proto-West Slavs are far Elbe, and push the East Slavs to the North, displacing the Balts and Balto-Finns in their drive north.
	By 600, I would suspect the culture map would start reflecting the position in Vanilla CK3.
	The Vlach remain an open and contentious question, but the CK3 systems really don't model peripatetic peoples living contemporaneously with settled populations; given Vanilla CK3 gives them pastoralists, they probably were an amalgamation of Romance speakers with nomads during the Pannonian Avar period.
	
	Beyond the reaches of the Balts, the situation in Scandinavia is also problematic on the CK2 map. Actual arrival of the Finns in Finland is dated to at the earliest, when WtWSMS \cite{LaplandSaami} (See \cite{SaamiMap} for a good breakdown of extents of Saami prehistorical settlement in Norway/Sweden).
	Before then, there was a likely Paleo-Laplandic speaking peoples in the region, who were eventually subsumed into the Saami; this, however, occurred well before the start of the mod (or at most, the Paleo-Laplandics died out before the Viking Era). 
	For simplicity, the author proposes we simply lump the Paleo-Laplandics in with the Saami and not represent their possible surivial into the mod period.
	
	Regardless, the Balto Finns remained to the South of Finland until around the VIth century, although the development of the Balto-Finnic civilization was underway, with linguistic splits into Estonian, Finnish, and Karelian occurring between 600 Bc and AD 150 and only later splitting off \cite{LaplandSaami,DiversificationOfProtoFinnic}.
	\cite{DiversificationOfProtoFinnic} argues the uniform penetration of Slavic Christian loanwords into Balto-Finnic languages indicates the Balto-Finnic languages were still relatively close geographically and linguistically in the VIIIth century, an argument the author finds a strong indicator that the push of the Finns into Finland came with the pressure of the Slavic incursions into what would be the land of the Chuds, historically.
	Either way, there should be no Proto-Finns in Finland before the VIth century at the earliest, and they slowly displace the Saami in Finland as time goes on.
	
	The author think this narrative is more cohesive and matches available literature on barbaria better than the CK2 WtWSMS game history, which leans too heavily on Florin Curta’s work; his work is far from universally accepted and in CK2 WtWSMS, puts the Sklaveni and Antes near the Roman border in 476, where they were not even attested until the Early VIth Century, at least 50 years after the Western Part of the Empire fell. 
	
	\section{Timeline and Evolution}
	\label{sec:timeline}
	
	In this section, note that the "political control" doesn't always imply direct control \textit{per se}; it more implies where a non-Roman polity of some sort existed within Roman lands or varying degrees of control in Barbaria.
	The precise modeling of how to handle things such as semi-landed federates is a separate discussion outside the scope of this document, so view the various polities on the control maps as notional.
	
	Additionally, the situation in the Middle and upper Danube regions of Barbaria during the Vth century is mostly poorly attested, from what the author can determine, so lacking additional input, the Marcomanni, Suebi, and Buri remain in the region post-migration, and should be modeled as individual chiefdoms. 
	By the mid-to-late Vth century, we have a slightly better picture of what's going on along the Danube.
	
	Finally, there's a question of what is "Suebi"; historically, this term encompassed a wide number of peoples WtWSMS treats as separate cultures, such as Angles, Reudingians, Thuringians, Suebi, Bucinobantes, and Alamannians.
	Barbarian identities were not nearly as fixed as modern identites, nor even those of the Medieval era. 
	
	\subsection{The Situation in AD 350}
	\label{sec:timeline:subsec:350}
	
	Working from the previous Archelogical Complex map, we can begin to fill in a \textit{potential} culture map.
	So much of Barbarica remains unknown that besides cultures closely bordering Rome (Burgundians, Marcomanni, Quadi, Iazyges, Vandals, Taifals, Goths), we can only guess at how things are arranged.
	As such, this can be a point of contention and the author encourages others to make counterarguments, and there were overlapping notions of identity; one could be both Alamanni and Suebi, so this presents a conundrum.
	To resolve that, the author proposes that the "Suavi" identity be linked to the Suebics of the Kingdom of the Suebi in Hispania and the Suebi identity belong to the post-405 migration Suebic Danubian peoples, especially Hunimund's Suebic Kingdom in the 450s.
	
	\begin{figure}[h!]
		\centering
		\includegraphics[width=1.00\textwidth]{./images/cultures_350AD.png}
		\caption{Culture proposal ca. AD 350. Squares indicate minorities (two for large, one for small).
			\newline\tiny The other bloc is a block of unknowns; probably a Germano-Dacian-Sarmatian mix, possibly Buri or Dacian Buris or Victohali Germans}
	\end{figure}
	
	\begin{itemize}
		\item Rygir:\newline
		The Rygir are a giant unknown, but we postulate they simply migrated south from Pommerania; this is the most parsimonious path they could have taken, especially when the Marcomanni et al flee the Huns.
		
		\item Oium:\newline
		The Skirians Heruli, and Gepids were all put adjoining the Goths.
		It is known that the former two raided the Black Sea (the Heruli with the Goths), and the Skirians would raid occasionally with the Dacians, so this seems to be a nice compromise.
		We put the Gepids at the source of the Dniester because it is mountainous (per \textit{Getica}'s reference to their lands) but puts them in a position to replaces the Vandals after the Vandals migrate.
		
		\item Buri:\newline
		The Buri were listed by Tacitus as a Suebic Tribe but as a Lugii Tribe by Ptolemy.
		That mountainous location puts them just beyond the reach of the later Przeworsk \cite{HeatherEmpiresAndBarbarians,PrzeworskHistory}, but also next to the Marcomanni and Quadi, and would allow the Buri to be considering part of either group
		
		\item Taifals:\newline
		The Taifals are put in Oltenia where they settled in the late IIIrd Century.
		This puts them next to the Thervingi Goths (going by the East/West split ob the Dniester, which is a popular but not proven identification of the two), which their were tightly linked until the 370s.
		\item Burgundians:\newline
		The Burgundians, of course, remain near the Rhine border, despite them being an East Germanic group.
		
		\item Vandals:\newline
		The Vandals had a Kingdom at the upper reaches of the Tisza at least into the IVth century; the Prezeworsk late period had a part in the Upper Tisza at the time \cite{PrzeworskHistory}.\newline
		\textit{Getica} claims that the Vandals lost a battle with the Goths and migrated into Pannonia for 60 years.\newline
		Stilcho was of partly Vandalic descent as well, so there is probably some truth to the settlement at least.\newline
		For simplicity, probably make the Silingi keep the Tisza upper basin and the Hasdingi minorities in Pannonia; this keeps them in close contact and makes the later  migration make more logical sense even if Silingi gave their name to Silesia.
		
		\item Suebi:
		The Marcomanni and Quadi were still attested to until the turn of the Vth Century, so they remain. 
		For simplicity, they are one culture, separate from the other Suebians and are on the decline, getting muscled out by the new "Suebian" culture.
		
		\item There is very little literature LT-Rascek can find for the Lebus, Gustow, and Denziner.
		More sources would be appreciated (2025-01-23)
	\end{itemize}
	
	To the east lie the Huns, who are about to assault the Alans and Goths...
	
	\newpage
	\subsection{The Situation in AD 395}
	\label{sec:timeline:subsec:395}
	By 395, a number of important events had occurred, which started reshaping the cultural situation:
	
	\begin{itemize}
		\item Huns:\newline
		The Huns began their assault on the Alans and Goths, precipitating several migrations within Barbaria
		Huns were probably still disorganized or otherwise subordinating Dacia
		So the Hunnic polity is probably disorganized, but settling in Dacia \cite{OttoHuns}.
		The Huns are \textit{not} a unified polity at this point; there are records of various Hunnic rulers and raids, but only Uldin gets a mention, and some Hunnic forces enter into a federate agreement with the Romans near the Hungarian Plain \cite{OttoHuns}.
		Uldin's Huns carry out their raids of the Balkans in 395, likely near the Danube
		In 395, in the East, a large Hunnic host assaults over the Caucus while another crosses the Danube (likely Uldin's)
		
		\item Akatziri:\newline
		The Akatziri were both referred to as Scythian and Hunnish; the separate identity (and short time scale) would imply Scythian while their union with the Huns would imply a Hunnish connection.
		They were a separate people, however, and had their own king, Curidachus \cite{PLRE_Vol2} who then dealt with Attila. 
		Priscus refers to them as both Scythian and Hunnic \cite{CambridgeHistoryEarlyInnerAsia}\specificCite{pg. 191}, but beyond that we have little.
		Either they were the remnant Alans who converted to Huns, the Eastern Branch of the Huns, or "an indomitable forest people \cite{CambridgeHistoryEarlyInnerAsia}\specificCite{pg. 191}".
		
		\item Bolghars:\newline
		The Bolghars aren't referenced before 480 in Europe, so likely beyond the Volga prior to 480 \cite{KimHuns}.
		
		\item Greuthungi:\newline
		Subordinated by the Huns, with some groups like Odotheus fleeing the Huns and trying to break into the Empire
		The main component remains under control of Alatheus and Saphrax.
		Radagaisus' component is probably pushed into central Dacia around this period; makes Cacualand (Ammianus Marcellinus) likely in Central/East Transylvania.
		The proto-Amalings were almost certainly still under Hunnic domination, close to the main Hunnic body.
		
		\item Therivingi:\newline
		After the Huns arrive, they subordinate Gutthiuda and push the Thervingi into Roman territory, building some sort of polity under nominal Roman domaince
		The Goths achieved a strategic victory in the War of 376-82, allowing them to colonize Roman lands south of the Danube and remain under \cite{HeatherEmpiresAndBarbarians}\newline
		
		\item Skirians:\newline
		The Skirians make their reentry into the historical record in the 380s following some raids onto Roman land, as part of a combined Dacian/Skirian/Hunnic raid.\newline
		This would imply that the Skirians needed to be near both the Huns and Dacians, making their position more clear.
		
		\item Taifals:\newline
		The Taifals became federates outside of Rome, then lost and were resettled into Italy, Gaul, and Phyrgia \cite{WolframHistoryOfTheGoths} \specificCite{Ch. 3, §1, pg. 123}
		
		\item Alans:\newline
		We put the Alans in the previous "Other" column as that puts them near the Vandals in time for their migration to Gaul, which will also pull in the remaining Marcomanni and some of the Suebi.\newline
		The Alans would have blended in with the Iazyges, who were now being referred to as Sarmatians again.\newline
		
		\item Marcomanni:\newline
		The last definitive communication with the Quadi/Marcomanni takes place between Ambrose of Milan and Fritigil ca. 397, so they certainly continued for several years later at a minimum, probably as a migrating host under Fritigil and another part in Marcomannia proper.
	\end{itemize}
	
	\newpage
	\begin{figure}[h!]
		\centering
		\includegraphics[width=0.85\textwidth]{./images/cultures_395AD.png}
		\caption{Culture proposal ca. AD 395. Squares indicate minorities (two for large, one for small).}
	\end{figure}
	
	\begin{figure}[h!]
		\centering
		\includegraphics[width=0.85\textwidth]{./images/political_control_395.png}
		\caption{Political Proposal, AD 395}
	\end{figure}
	
	\newpage
	\subsection{The Situation in AD 405}
	\label{sec:timeline:subsec:405}
	
	\begin{itemize}
		\item Huns:\newline
		The Western Hunnic portion takes shape; the proto-Ostrogoths between the Carpathians and Dneister, the Gepids.
		The Skirians and Herlui were still part of the confederation, but mentioned more than the Greuthungi.
		Overall western control probably spread as far as the Heruli Kingdom and most of Dacia, giving the Huns an admixture of Slavs, Dacians, Gepids, Heruli, Skirians, and Iazyges to control.
		Possibly the Huns were still in two groups split by the Dneister; the Pontic and Dacian Huns of the period don't seem to be very coordinated.
		Probably have the Huns settling in the Great Hungarian Plain ca. 410s.
		Most likely, the Huns controlled the Middle and Lower Danube to varying degrees, but the Huns were still not a unified force even in the West.
		The author presumes this control/influence stretched as far as Noricum, in line with the general dearth of information in the region prior to the collapse of the Hunnic Empire.
		
		\item Bolghars:\newline
		The Bolghars aren't referenced before 480 in Europe, so likely beyond the Volga prior to 480 \cite{KimHuns}.
		
		The Akatziri were both referred to as Scythian and Hunnish; the separate identity (and short time scale) would imply Scythian while their union with the Huns would imply a Hunnish connection.
		For the purposes of the mod and lacking much definite attestation, we should make them a Hunnic branch.
		
		\item Heruli \& Sciri:\newline
		Lacking other attestation (and the fact that the Sciri and Dacians raided Thrace a few decades ago), they should remain between the Carpathians and Dniester, putting them in position for eventual migration to Pannonia along with the Amalings in the mid-Vth Century.
		
		\item Greuthungi:\newline
		Radagaisus' component attempts its ill-fated invasion of Roman territory, opening the region for Gepid colonization.
		The Amalings coalesced in the Dneister swamp region, which remains ill-suited for Hunnic occupation.
		The terminology for the Geruthungi begins to decline in the early Vth Century, which may signify it as a regional/directional term \cite{WolframHistoryOfTheGoths}.
		So most likely the Huns started consolidating the Goths under them in what was formally Therivingi lands, which would explain why Radagaisus would have fled.
		The Amalings probably began coalescing as the primary Goths under the Huns during this period and were almost certainly in communication with the Moesian Gothic contingent (enough for them to be considered a single people still)
		
		\item Therivingi:\newline
		Under Alaric I, the Therivingi as starting to consolidate into a new identity and have become a migratory host, leaving behind the lesser Goths, camping somewhere in Pannonia.
		This presages the eventualy loss of Pannonia to various non-Romans, probably when Attila rises to prominence.
		
		\item Vandals:\newline
		The Vandals are on the move, migrating and sacking Gaul and thus have no map representation at this time.
		
		\item Alans:\newline
		The Alans joined with the Vandals, but Goar and Respindal split, with Goar supporting the Romans and Respindal continuing the Alanic march.
		Goar's initial settlement would be in the north near Soissons initially \cite{BachrachAlans}\specificCite{Ch. 2}.
		This settlement means, in part, they likely played a role in boxing in the Franks by virtue of distance.
		
		Respindal's Alans would continue onward...
		
		\item Gepids:\newline
		With the flight of the Vandals, the Gepids start building their kingdom in Dacia, replacing the Vandals and filling in where the Alans left.
		
		\item Subei:\newline
		The remnant Suebi remain adjoining the Alamanni, with the bulk of what was in Bohemia about to migrate into Gaul.
		
		\item Buri \& Marcomanni:\newline
		The Buri and Marcomanni more-or-less disappear from the historical record mostly.
		The Buri end up settling in Galacia, giving some place names, while the Marcomanni pretty much disappear entirely.
		For the purposes of mod mechanics, there's remnant Suebic cultures along the Middle Danube for lack of other attestation before the mid-Vth century.
		
		\item Burgundians:\newline
		The Burgundians are nearly ready to establish a kingdom across the Rhine.
		
		\item Franks:\newline
		The Franks repulsed the Vandal/Suebi/Alan host, which is positioning itself to cross into Gaul.
		
		\item Rygir:\newline
		The Rygir are scarcely attested; they \textit{may} have been attested in the early IVth, but that is not clear.
		It is more clear that they become part of the Hunnic Confederation under Attila.
		With the Suebi migrating out, they fill up what will become Bohemia, at least the northern portion; this will put them directly in line with eventual Hunnic expansions that will united (Danube) Suebi, Rugii, Therivingi, and other Germanics as part of Attila's forces.
		
		\item Langobards:\newline
		A precise timeline is not available, but the legendary route of the Lombards is known.
		Scoringa (Elbe shores?), Mauringa, Golanda (Oder?), Anthaib, Banthaib, Vurgundaib, Rugiland.
		In Vurgundaib, the Bulgars (maybe Huns) would subordinate the Langobards.
		If Vurgundaib is the old land of the Burgundians, this would imply that they migrated into the old Burgundian lands just prior to Ruigland, would fill in Rugiland as the Rygirs departed.
		If Vurgundaib is in the mountains to the north of Bohemia, then having the Langobards migrate into the old Rygir lands would make those Anthaib/Banthaib.
		So the Langobards get placed in the old Rygir lands, perhaps encroaching slightly.
		
		\item Iazyges:\newline
		The Iazyges as a term died out in Roman literature by this point, but a Sarmatian polity remained in their old lands, for which the Iazyges remain the representative, expanding a bit as the Alans migrated away.
		
		\item Slavs:\newline
		The Slavs still exist outside the historical record; for mod purposes, they end up back-filling most territory in which East Germanics migrate away from where they are not attested.
	\end{itemize}
	
	\newpage
	\begin{figure}[h!]
		\centering
		\includegraphics[width=0.85\textwidth]{./images/cultures_405AD.png}
		\caption{Culture proposal ca. AD 405. Squares indicate minorities (two for large, one for small).}
	\end{figure}
	
	\begin{figure}[h!]
		\centering
		\includegraphics[width=0.85\textwidth]{./images/political_control_405.png}
		\caption{Political Proposal, AD 405}
	\end{figure}
	
	\newpage
	\subsection{The Situation in AD 415}
	\label{sec:timeline:subsec:415}
	The Danubian Suebic culture evolves from the remaining Marcomannic and Buri cultures along the Danube.
	
	\begin{itemize}
		\item Huns:\newline
		Probably have the Huns settling in the Great Hungarian Plain ca. 410s.
		Uldin's death leads to Hunnic raiding dying down a bit.
		Most likely, the Huns controlled the Middle and Lower Danube to varying degrees, but the Huns were still not a unified force even in the West.
		Control in the West is beginning to consolidate as Octar and Rugila rule, possibly in an East/West division.
		Alternatively, they divided the west.
		
		\item Bolghars:\newline
		The Bolghars aren't referenced before 480 in Europe, so likely beyond the Volga prior to 480 \cite{KimHuns}.
		
		\item Skirians and Heruli:\newline
		As before, still stuck beyond the reaches of Rome.
		
		\item Greuthungi/Ostrogoths:\newline
		The Greuthungi/Ostrogoths remain mostly unattested from the arrival of the Huns to the rise of Atilla, as the Gruthungi signifier started melting away.
		
		\item Rygir:\newline
		The Rugii remain shadowy, not really showing up in records yet until later, but were part of the Hunnic Empire by Atilla's time.
		
		\item Therivingi/Visigoths:\newline
		The Therivingi under Alaric continued their path in Rome, besieging it 408 and receiving ransom.
		Politicking with the Romans, he eventually is attacked by Sarus, which makes Alaric end negotiations and sack Rome, afterwards dying somewhere in Southern Italy.
		Under his successor Athaulf, they managed to secure the initial Visigothic Kingdom on the Garonne, settling as federates.
		
		\item Vandals:\newline
		The Vandals exist in two groups:  Hasdingi in North Hispania and Silingi in South Hispania.
		Both are Roman Federates, and Rome is about to turn on them, as they raid Hispania.
		
		\item Alans:\newline
		Respindal's (now Attaces') Alans have settled in Southern Hispania, surrounding the Silingi Vandals.
		The Visigoths will soon turn on them and expel the Vandals and Alans from Hispania.
		
		Goar's Alans are flexing their muscles, with the Burgundians and their puppet Jovinus.
		Proximity almost certainly played a role for both supporting Jovinus and collaberating with the Burgundians \cite{BachrachAlans}\specificCite{Ch. 2}.
		
		A contingent from Respindal's group broke form their Therivingi/Visigothic allies in 414 during the siege of Bazas and received lands between Toulouse and Nabarone to settle. \cite{BachrachAlans}\specificCite{Ch. 2}.
		
		\item Gepids:\newline
		The Gepids remain subordinate to the Huns, although not yet prominent.
		
		\item Subei:\newline
		The Suebian Kingdom in Hispania is established in 409.
		The Hasdingi Vandals muscle into the East and thus probably keep the Suebi to the coast.
		
		The Marcomanni and Buri settle in various parts of their kingdom as well.
		
		Along the Danube, there remains a nucleus of Suebi/Marcomanni/Buri that'll become Hunimund's Kingdom in the 450s \cite{PLRE_Vol2}
		
		There are still references to the Juthungi as late as the 430s, so somewhat formidable Suebic polity should exist between the Alamanni and Bohemia.
		
		\item Burgundians:\newline
		In 411, the Burgundians set up Jovinus in conjunction with Goar, settling on the left bank of the Rhine, between Lauter and Nahe rivers.
		There they will raid Gaul for decades before being brought heel in 436.
		
		\item Franks:\newline
		Romans are beginning to loose control with usurpers, but expansion past the core Salian and Ripuarian regions is in the future.
		
		\item Langobards:\newline
		Following the migration to Anthaib and Banthaib, the Lombards reached Vurgundaib.
		If Vurgundaib is the old land of the Burgundians, this would imply that they migrated into the old Burgundian lands just prior to Ruigland, would fill in Rugiland as the Rygirs departed.
		There, they would fight "Bulgars" (Huns?), so they became at least peripherally part of the Hunnic Empire.
		Debatable if they became a part of it, but old Burgundian lands would leave them almost in a direct path from Dacia to Gaul north of the Danube in time for Chalons.
		So they should migrate to Vurgundaib prior to the rise of Atilla, so NLT the 420s.
		
		\item Iazyges:\newline
		Really called Sarmatians now, were likely admixing with the Alans and Huns and such.
		Still retained a semi-separate identity as late as the end of the Vth Century (Babai et al)
		
		\item Slavs:\newline
		The Slavs still exist outside the historical record; for mod purposes, they end up back-filling most territory in which East Germanics migrate away from where they are not attested.
	\end{itemize}
	
	\newpage
	\begin{figure}[h!]
		\centering
		\includegraphics[width=0.85\textwidth]{./images/cultures_415AD.png}
		\caption{Culture proposal ca. AD 415. Squares indicate minorities (two for large, one for small).}
	\end{figure}
	
	\begin{figure}[h!]
		\centering
		\includegraphics[width=0.85\textwidth]{./images/political_control_415.png}
		\caption{Political Proposal, AD 415}
	\end{figure}
	
	\newpage
	
	\subsection{The Situation in AD 425}
	\label{sec:timeline:subsec:425}
	
	\begin{itemize}
		\item Huns:\newline
		The Huns raid again under Ruga, reaching Constantinople, forcing tribute and fighting for the Romans in North Africa.
		Aetius marched into Italy with the Huns, reconciled with the East, and marched back.
		\cite{OttoHuns} argues that in 427, the Romans broke their alliance with the Huns and marched into Pannonia, so probably controlled or settled in Pannonia Valeria during the 410s-420s.
		
		\item Bolghars:\newline
		The Bolghars aren't referenced before 480 in Europe, so likely beyond the Volga prior to 480 \cite{KimHuns}.
		
		\item Skirians and Heruli:\newline
		Both remain linked to the Huns, but remain relatively unattested before Atilla.
		
		\item Greuthungi/Ostrogoths:\newline
		The Eastern Goths or remnant Goths are likely disorganized in Thrace and the Amals are likely ruling over the Goths beyond the Roman sphere.
		They more-or-less drop from recorded history until near the end of the Hunnic empire, although they are likely being internally migrated.
		
		\item Rygir:\newline
		Beyond being listed as part of Attila's forces in the 450s, little is told of the Rygir during this period.
		
		\item Therivingi/Visigoths:\newline
		The end of the Gothic War in Spain sees the Vandals and Alans expelled from Hispania, with the Visigoths starting to expand their influence and control into Aquitania Secunda \cite{WolframHistoryOfTheGoths}.
		
		\item Vandals:\newline
		The Silingi and Alans were routed in the late 410s, united under the Hasdingi in the person of Gunderic, who fled to Baetica.
		Wining the Battle of Tarraco, Gunderic is plundering Hispania, the Balearics, and Mauritania, sacking Sevilla and Cartagena this year, based in Baetica.
		So probably best modeled as a migrating tribe again, pulling away most of the Vandals and Alans in Hispania under his banner.
		
		\item Alans:\newline
		The Alans of Hispania have joined with the Vandals, now united under the Hasdingi banner, plundering Hispania and Mauretania and preparing to leave Hispania.
		
		Goar's Alans remain relatively loyal and sedate during the period.
		
		\item Gepids:\newline
		The Gepids remained under Hunnish domination.
		
		\item Iazyges:\newline
		The Iazyges remained under Hunnish domination.
		
		\item Subei:\newline
		With the Vandals gone, the Suebi are now uncontested in Galacia, and are in the process of solidifying their hold on Galacia.
		Hermeric is still king and will continue until 438.
		
		\item Burgundians:\newline
		The Burgundians still have their kingdom across the Rhine, with the capital in Borbetomagus.
		They will not be brought to an end until 436-7.
		
		\item Franks:\newline
		The Franks reamin relatively quiet; there was not a great push until the 440s to consolidate the North.
		
		\item Langobards:\newline
		Following the migration to Anthaib and Banthaib, the Lombards reached Vurgundaib.
		If Vurgundaib is the old land of the Burgundians, this would imply that they migrated into the old Burgundian lands just prior to Ruigland, would fill in Rugiland as the Rygirs departed.
		There, they would fight "Bulgars" (Huns?), so they became at least peripherally part of the Hunnic Empire.
		Debatable if they became a part of it, but old Burgundian lands would leave them almost in a direct path from Dacia to Gaul north of the Danube in time for Chalons.
		So they should migrate to Vurgundaib prior to the rise of Atilla, so NLT the 420s.
		
		\item Slavs:\newline
		The Slavs still exist outside the historical record; for mod purposes, they end up back-filling most territory in which East Germanics migrate away from where they are not attested.
	\end{itemize}
	
	\newpage
	
	\begin{figure}[h!]
		\centering
		\includegraphics[width=0.85\textwidth]{./images/cultures_425AD.png}
		\caption{Culture proposal ca. AD 425. Squares indicate minorities (two for large, one for small).}
	\end{figure}
	
	\begin{figure}[h!]
		\centering
		\includegraphics[width=0.85\textwidth]{./images/political_control_425.png}
		\caption{Political Proposal, AD 425}
	\end{figure}
	
	\newpage
	
	\subsection{The Situation in AD 435}
	\label{sec:timeline:subsec:435}
	
	\begin{itemize}
		\item Huns:\newline
		Ruga is still king of the west, and possibly his brother Octar as king of the East.
		Octar died in 430 fighting the Burgundians, leaving a gap in the history; either Atilla succeeded him in the East \cite{CambridgeHistoryEarlyInnerAsia} \specificCite{pg. 188} but Maenchen-Helfen argues Ruga became sole ruler \cite{OttoHuns}.
		The Treaty of Margus was signed in 435 by Rome as well as both Bleda and Atilla.
		
		To keep things simple for CK3 succession mechanics, we should go with the later, barring some method of modeling the complexities of steppe confederations.
		Hunnic influence was probably extending as far as the Elbe, seeing the West Huns attack the Burgundians in 430 and the fat that the Thuringii would become a part of Atilla's confederation.
		Aetius ceded Pannonia north of the Sava sometime before Attila's rise to power; the mid-430s is good, as this gives the Huns a large plain \& economic base (e.g., room for horses) to assert themselves, which is probably why Attila would become such a threat in the 440s-450s.
		
		\item Bolghars:\newline
		The Bolghars aren't referenced before 480 in Europe, so likely beyond the Volga prior to 480 \cite{KimHuns}.
		
		\item Skirians and Heruli:\newline
		Both remain linked to the Huns, but remain relatively unattested before Atilla.
		
		\item Greuthungi/Ostrogoths:\newline
		The Eastern Goths or remnant Goths are likely disorganized in Thrace and the Amals are likely ruling over the Goths beyond the Roman sphere.
		They more-or-less drop from recorded history until near the end of the Hunnic empire, although they are likely being internally migrated.
		
		\item Rygir:\newline
		Beyond being listed as part of Attila's forces in the 450s, little is told of the Rygir during this period.
		
		\item Therivingi/Visigoths:\newline
		The Visigoths remain under Theodoric I.
		The peace with Rome remains uneasy, and the Visigoths regularly siege Arles; exact reasons are lost to history, but it never really led to a seizure of land or booty \cite{WolframHistoryOfTheGoths}\specificCite{Ch. 4, §1, pg. 175}.
		So no real exapnsion
		
		The Visigoths are about to terminate their Federate treaty in 436 and launch a war, which they will loose and become semi-loyal federates again.
		
		\item Vandals+Alans:\newline
		The Vandals departed Hispania in 429 with most of the Alans in Hispania, and by 435, the Vandals took parts of Mauretania, with Hippo Regius as his capital, along with teh Balrics, Sardinia, and Corsica. 
		
		\item Gallic Alans:\newline
		Goar's Alans remain sedate during the period, but will soon be resettled by Aetius in the 440s to Orelans \cite{BachrachAlans}\specificCite{Ch. 2}.
		
		To the South, the Valentinois Alans are about to settle ca. 440 under Sambida, after which they will rebel under Beogar in 457 before becoming a non-entity \cite{BachrachAlans}\specificCite{Ch. 2}.
		
		\item Gepids:\newline
		The Gepids remained under Hunnish domination.
		
		\item Iazyges:\newline
		The Iazyges remained under Hunnish domination.
		
		\item Subei:\newline
		Hermeric is still king and will continue until 438; his successor Rechila will initiate an attempted conquest of Hispania, but not until 438-448
		
		\item Burgundians:\newline
		The Burgundians still have their kingdom across the Rhine, with the capital in Borbetomagus.
		They will not be brought to an end until 436-7.
		
		\item Franks:\newline
		The Franks reamin relatively quiet; there was not a great push until the 440s to consolidate the North.
		
		\item Langobards:\newline
		Still in Vurgundaib prior to the Rygir moving into Noricum.
		
		\item Slavs:\newline
		The Slavs still exist outside the historical record; for mod purposes, they end up back-filling most territory in which East Germanics migrate away from where they are not attested.
	\end{itemize}
	
	\newpage
	
	\begin{figure}[h!]
		\centering
		\includegraphics[width=0.85\textwidth]{./images/cultures_435AD.png}
		\caption{Culture proposal ca. AD 435. Squares indicate minorities (two for large, one for small).}
	\end{figure}
	
	\begin{figure}[h!]
		\centering
		\includegraphics[width=0.85\textwidth]{./images/political_control_435.png}
		\caption{Political Proposal, AD 435}
	\end{figure}
	
	\newpage
	
	\subsection{The Situation in AD 445}
	\label{sec:timeline:subsec:445}
	The mid-Vth century is when things become hairy along the Danube; this is when we start seeing the nucleus of the various Germanic kingdoms in the Middle Danubian region.
	Pannonia is devastated, cities razed, and population displaced by Hunnic Empire forming along the Danube.
	The Ostrogoth color is used, but the culture should not exist yet, being a game-mechanical byproduct of the migration of the eastern Goths to Italy in the 480s.
	
	\begin{itemize}
		\item Huns:\newline
		With Ruga having died, Attila dn Bleda rule the Huns, probably as separate wings with one superior to the other; Kim thinks Bleda was the superior King \cite{KimHuns}.
		The Treaty of Margus was singed in 435 forcing Rome to grant tribute to the Huns, and Hunnic attention turned towards the Caucuses and Persia for a few years.
		They did assist the Romans somewhat, destroying the Burgundian Kingdom in 436 and launching raids on the Visigoths in the late 430s.
		
		By the 440s, the Persia endeavor had failed and the Huns turned their attention to Rome again, launching devastating raids on the Balkans.
		Following the devastation, some diplomatic maneuvering between the Huns and Rome led to an ease of tensions.
		Authors disagree as to the timing, either in 443-5 or 445-7; Kim argues for this occurring only after Attila achieved complete dominance.
		The author thinks its the later, with Bleda's death occurring around 447; this would better explain why the Western (Attilid) Huns would stop pressing Rome as Attila decides to handle Bleda and bring all the Huns under a single banner.
		
		\item Bolghars:\newline
		The Bolghars aren't referenced before 480 in Europe, so likely beyond the Volga prior to 480 \cite{KimHuns}.
		
		\item Skirians and Heruli:\newline
		The Sciri and the Heruli probably migrate to the Middle Danube region around this time.
		
		For the Sciri, see: \cite{WolframHistoryOfTheGoths,ToddEarlyGermans} \specificCite{264}; \specificCite{pg. 225}\footnote{"A group of three graves at Bakodpuszta can reasonably be associated with the family of Edica himself. Not far away, in the Sarviz marshes, a magnificent shield adorned with almost a kilogram of gold and over a thousand garnets has also been linked to the ruling house, possibly with Edica himself."}
		
		The Heruli position is more mysterious, but somewhere north of the Danube, so probably where it was in CK2 works well enough.
		
		\item Greuthungi/Ostrogoths:\newline
		\cite{WolframHistoryOfTheGoths}\specificCite{pg. 261} outlines Amal Goth settlement from the Southwest corner of Lake Balton to the approaches of Sirmium (including Mursa-Osijek), as a united polity with two additional sub-rulers.
		
		\item Rygir:\newline
		Beyond being listed as part of Attila's forces in the 450s, little is told of the Rygir during this period.
		
		\item Therivingi/Visigoths:\newline
		The Visigoths are still a federate kingdom, although the grip of Imperial power is weakening.
		Theodoric I controls the Visigothic state, and is in tensions between the Romans, Suevi, and Vandals.
		But in terms of settlement and political borders, it seems to have been unchanged.
		
		\item Vandals+Alans:\newline
		The Vandals departed Hispania in 429 with most of the Alans in Hispania, and by 435, the Vandals took parts of Mauretania, with Hippo Regius as his capital, along with teh Balrics, Sardinia, and Corsica. 
		
		\item Gallic Alans:\newline
		Goar's Alans are resettled by Aetius in 442 to Orelans \cite{BachrachAlans}\specificCite{Ch. 2} to deal with the Armoricans and the general banditry in Armorica.
		This, coincidentially, opens a gap for the Franks to expand into Northeast France.
		
		The Valentinois Alans settled ca. 440 under Sambida, after which they will rebel under Beogar in 457 before becoming a non-entity \cite{BachrachAlans}\specificCite{Ch. 2}.
		
		\item Gepids:\newline
		The Gepids remained under Hunnish domination.
		Primary settlement area is the Someș and Körös rivers, although it would probably shift further south with Slavic colonization of the mountains.
		
		\item Iazyges:\newline
		The Iazyges remained under Hunnish domination, although admixture with Huns seems likely as well; while Romans used the term Sarmatian for Babai's group, it was likely Hunnized in the Vth century.
		
		\item Subei:\newline
		Rechila succeeded his father in 438 functionally (Hemeric is still ill).
		He launched a series of wars of expansion, taking Merida in 439 and all of Lusitania and Baetica by 441.
		Hemeric dies in 441.
		Hydatius states that the Suebi took both Baetica and Carthaginenesis, but for mechanics reasons it is more accurate to show them as occupied or as a pair of wars against semi-independent Roman polities, or at least having Carthaginensis as a tributary state.
		
		In 446, the Romans will launch a counter attack with both Goths and Roman, which will be rebuffed.
		
		\item Burgundians:\newline
		The Burgundians were routed from their Rhineland Kingdom in 436 by a combined Hun/Roman force, complete with the destruction of Worms. 
		The Burgundians disappeared from history for a few years, but were then resettled in 443 in Sapaudia by Aetius.
		
		\item Franks:\newline
		With the movement of the Alans to Orelans, the Franks are free to press further south.
		There's no record connecting the two directly, but it seems to be a logical transition.
		Actual settlement probably wasn't expanding \textit{per se}, but political control almost certainly was; if not, then the Franks extended their rule in the chaos of the Hunninc invasion in the 450s.
		
		\item Langobards:\newline
		Still in Vurgundaib prior to the Rygir moving into Noricum.
		
		\item Slavs:\newline
		The Slavs still exist outside the historical record; for mod purposes, they end up back-filling most territory in which East Germanics migrate away from where they are not attested.
	\end{itemize}
	
	\newpage
	
	\begin{figure}[h!]
		\centering
		\includegraphics[width=0.85\textwidth]{./images/cultures_445AD.png}
		\caption{Culture proposal ca. AD 445. Squares indicate minorities (two for large, one for small).}
	\end{figure}
	
	\begin{figure}[h!]
		\centering
		\includegraphics[width=0.85\textwidth]{./images/political_control_445.png}
		\caption{Political Proposal, AD 445}
	\end{figure}
	
	\newpage
	
	\subsection{The Situation in AD 455}
	\label{sec:timeline:subsec:455}
	The Huns have passed the zenith of their power and their control is broken.
	Pannonia is devastated, and most of its settlements abandoned (or, at least, the non-slave/peasant population flees south).
	The Vandals sack Rome in 455, setting off the chain of events that will politically lobotomize the Western Empire.
	Ostogothic culture color is used for clarity, but Ostrogothic culture would not form until the final migration of the Ostrogoths to Italy as per new CK3 landless mechanics.
	
	\begin{itemize}
		\item Huns:\newline
		Under Attila, the Huns reach the zenith of their power in 451, invading and devastating Gaul, before being rebuffed at Chalons.
		He would be planning on making the Eastern Roman Empire submit in 453 when he died.
		The rapid succession between Attila's sons probably contributes to the disintegration of the Huns in Europe.
		
		Tuldila's host probably broke from the main Hunnic host ca. 458, leaving Dengzich with his polity in south Ukraine/south Moloova.
		
		Following the Battle of Nedao, the Hunnic empire has been effectively expelled from most of Europe, except possibly along the lower Danube as a united polity under the Attilids.
		Continued references to Hunnic peoples and groups into the VIth century implies that a number of Huns may have survived as smaller, disunited groups in Europe for decades following, but the Huns now are fundamentally split.
		Their Barbarian vassals engage in a series of power plays to determine dominance in Pannonia.
		
		\item Bolghars:\newline
		The Bolghars aren't referenced before 480 in Europe, so likely beyond the Volga prior to 480 \cite{KimHuns}.
		
		\item Skirians, Heruli, Danube Suebi:\newline
		The three small Barbarian Kingdoms are in the middle of fighting for dominance in the vacuum the Huns left.
		These power plays won't readily resolve until the late 460s, when Suebi and Sciri will flee for greener pastures after loosing to the Ostrogoths.
		
		\item Greuthungi/Ostrogoths:\newline
		The two remaining Gothic components are likely to shift towards Pannonia and Moesia, the remaining centers of Greuthungi/Gothic civilization near the Romans. 
		
		\item Rygir:\newline
		Beyond being listed as part of Attila's forces in the 450s, little is told of the Rygir during this period.
		
		\item Therivingi/Visigoths:\newline
		During the Battle of Chalons, Theodoric I died, leaving his son Thorismund in control fo the Visigoths; the Visigoths played a crucial role in the battle.
		
		The Western Goths are still mostly loyal federates after the earlier rout in Tolosa, but will be called upon to wrest Baetica in the next few years.
		
		Beyond that, they waged some warfare against the Orleans Alans, but did not decimate them \cite{BachrachAlans}.
		
		\item Gallic Alans:\newline
		The Orleans Alans held out against a Hunnic siege and played a role in the battle of Chalons.
		Their control shrinks as the Visigoths push them across the Loire, but they were not completely decimated \cite{BachrachAlans}.
		
		Sambida's Valentois Alans serve as a bulwark against the Visigoths and peasant revolts, which Bachrach links to teh Visigoth-linked Alans \cite{BachrachAlans}.
		They will play their role faithfully under Sambida but revolt under Beogar and threaten Italy before Beogar is killed and the Alans dispersed after 457 \cite{BachrachAlans}.
		
		\item Gepids:\newline
		With the Battle of the Nedao River, the Gepids become one fo the primary powers in the Pannonian region; Jordanes says (\textit{Getica}), and archeological evidence seems to point to northern settlement region \cite{GepidsInBalkans}.
		We know they did not occupy the region around Sirmium until the late Vth or early VIth centuries, so these regions were likely part of Sarmatian/Hunnic polities for the next two decades; we have attestations of Huns into the VIth and the Iazyges into the late Vth.
		
		\item Iazyges:\newline
		The Iazyges remained under Hunnish domination, although admixture with Huns seems likely as well; while Romans used the term Sarmatian for Babai's group, it was likely Hunnized in the Vth century.
		
		\item Subei:\newline
		The Suebi remain in control of Baetia (as well as possibly Cartagenesis) and Rome is about to turn its attention to the Suebi in 456.
		
		\item Burgundians:\newline
		Ricimer rises in the ranks of the Western Empire, which is probably from where the Burgundians could begin acquiring territory in Gaul (likely in conjunction with local elites).
		
		\item Franks:\newline
		Having participated in the Battle of Chalons and the Alans of Northwest Gaul gone, the Franks remain the premier power in the norther portions of Gaul.
		In legend, Merovech is the king of the Franks until 457/8, and Childeric I will assume control over the Salians in 458.
		More likely is that the Franks fill the void caused by the Huns and the general abandonment of northern Gaul after the Hunnic campaign.
		
		\item Langobards:\newline
		Still in Vurgundaib prior to the Rygir moving into Noricum.
		
		\item Slavs:\newline
		The Slavs still exist outside the historical record; for mod purposes, they end up back-filling most territory in which East Germanics migrate away from where they are not attested.
		In a nod to Florin Curta, we can have the Slavs start back filling where the Goths withdrew along the Danube, but they should remain under Hunnic domination until the late Vth Century.
		The mid-to-late Vth century is also the period where well-accepted Slavic archeological cultures (Prague-Korchak, \&c.) start to form, or at least no later than the early VIth century.
	\end{itemize}
	
	\newpage
	
	\begin{figure}[h!]
		\centering
		\includegraphics[width=0.85\textwidth]{./images/cultures_455AD.png}
		\caption{Culture proposal ca. AD 455. Squares indicate minorities (two for large, one for small).}
	\end{figure}
	
	\begin{figure}[h!]
		\centering
		\includegraphics[width=0.85\textwidth]{./images/political_control_455.png}
		\caption{Political Proposal, AD 455}
	\end{figure}
	
	\newpage
	
	\subsection{The Situation in AD 465}
	\label{sec:timeline:subsec:465}
	Ostogothic culture color is used for clarity, but Ostrogothic culture would not form until the final migration of the Ostrogoths to Italy as per new CK3 landless mechanics.
	
	\begin{itemize}
		\item Huns:\newline
		The Akatziri come under attack by Oghurs in the early 460s; as the author treats them as a main component of the Eastern Huns, this point probably represents the point when the Huns entered terminal decline.
		The Akatziri probably become functionally independent as the Huns shrink, under Ernak as Dengzich rules the western portions (probably from Moesia, maybe with the Dniester between them?)
		
		Moesia is probably abandoned by the Huns, given they are only linked in Roman records to the lowest part of the Danube by the end of the Hunnic Empire.
		
		The North Caucasian Huns (Khuni), attested to in Armenian missionary records, either formed from refugees of the Bolghar advance or were a component that became disconnected from the main body of the Huns by the Bolghar advance.
		The Kumyks consider themselves descendents of the Huns (at least in part), so there's no reason not to do so; Samandar makes a good a place as any to put them (in line with the Kumyks).
		
		Various other Hunnic peoples remain scattered across the Danube, without a unifying force.
		
		\item Bolghars:\newline
		The Akatziri come under attack by Oghurs in the early 460s; as the author treats them as a main component of the Eastern Huns, this point probably represents the point when the Huns entered terminal decline.
		At the very least, by 465, the Bolghars had overrun everything west of the Don, which puts them in position to negotiate with the 
		
		\item Skirians, Heruli, Danube Suebi:\newline
		The fight over control of Pannonia is gearing up, which is preparing for the Battle of Bolia in 469.
		The Rygir
		
		\item Greuthungi/Ostrogoths:\newline
		
		\item Rygir:\newline
		
		\item Therivingi/Visigoths:\newline
		
		\item Gallic Alans:\newline
		
		\item Gepids:\newline
		
		\item Iazyges:\newline
		
		\item Subei:\newline
		
		\item Burgundians:\newline
		
		\item Franks:\newline
		
		\item Langobards:\newline
		Still in Vurgundaib prior to the Rygir moving into Noricum.
		
		\item Slavs:\newline
		The Slavs still exist outside the historical record; for mod purposes, they end up back-filling most territory in which East Germanics migrate away from where they are not attested.
		In a nod to Florin Curta, we can have the Slavs start back filling where the Goths withdrew along the Danube, but they should remain under Hunnic domination until the late Vth Century.
		The mid-to-late Vth century is also the period where well-accepted Slavic archeological cultures (Prague-Korchak, \&c.) start to form, or at least no later than the early VIth century.
	\end{itemize}
	
	\newpage
	
	\begin{figure}[h!]
		\centering
		\includegraphics[width=0.85\textwidth]{./images/cultures_465AD.png}
		\caption{Culture proposal ca. AD 465. Squares indicate minorities (two for large, one for small).}
	\end{figure}
	
	\begin{figure}[h!]
		\centering
		\includegraphics[width=0.85\textwidth]{./images/political_control_465.png}
		\caption{Political Proposal, AD 465}
	\end{figure}
	
	\newpage
	
	\section{Culture Review}
	\label{sec:culture_review}
	
	With the hard part over, we can then start working on redefining the various aspects of cultures.
	The author wanted to make certain the aesthetics were more consistent and fix some things he found inconsistent or to better take advantage of CK3's new systems.
	
	\subsection{Germanic Heritages}
	\label{sec:culture_review:subsec:germanic_heritages}
	As mentioned previously, the arrangement of Germanic cultures in CK2 made little sense, and the porting of CK2 cultures to CK3 does not really resolve some of the underlying issues.
	CK3's more flexible culture system gives us an opportunity to do so.
	
	In particular, the North/West/Central/East split is a hodgepodge of CK2's Vanilla culture groupings and \textit{WtWSMS} cultures.
	But this approach has issues previously identified (like Suebic being West Germanic with Frankish, while other Suebic cultures are Central Germanic).
	
	With CK3's changes to heritages and the creation of Kulturbund's on top of that, we can revisit and overhaul the entirety of Germanic cultures.
	So the author proposes splitting the Germanic cultures into 6 separate heritages instead of 4:
	
	\begin{itemize}
		\item Scandzan (alias "North Germanic"):  North Germanic heritage of Late Antiquity, the Suiones/Sitones/Norse
		\item Istvaeonic (alias "Nordwestblock"): The Germanic cultures that evolved within the Nordwestblock with strong Celtic influence.
		The Cherusci, Frisii, Chamavi, and later Franks and Saxons.
		\item Cimbri (alias "Jutlandish"): The inhabitants of the Cimbrian Penninsula (today's Denmark).
		These tribes seem to be lumped in with Suebians by Tacitus but share features with the North Germanics as well, so function as an inbetween heritage.
		Includes the Eudoses/Jutes, Reudingians, Angli, Warni in Tacitus' \textit{Germania} XL all fall in this category.
		\item Suebic: For all the various Suebic tribes that aren't Cimbrian, including the Suevi, Alamanni, Marcomanni.
		\item Viscalan (alias "Prezworsk"): The cultures relating to the Przeworsk archaeological complex, including the Vandals.
		Other groups that are not Gothic or arose in Pommerania (which wasn't strictly Prezworsk) should fall under here as well.
		This includes the Wielbark/Gutones as they were before the Goths coalesced into a final form.
		Namesake is the rendering of the Vistula in Jordanes \textit{Getica}.
		\item Gothic (alias "Chernyakhov"): this is for all the cultures that seem to spring from the Chernyakhov complex, including the Goths, the Heruli, and the Skirians.
	\end{itemize}
	
	\begin{figure}[h!]
		\centering
		\includegraphics[width=1.00\textwidth]{./images/germanic_reorg2.png}
		\caption{Germanic Heritage Overhaul}
	\end{figure}
	
	In particular, this break provides a clear distinction between the Chernyakhov and Prezworsk groups of the East Germanic languages, which only briefly overlapped and for which there is an argument that the Goths arose \textit{in situ} on some fronts.
	This also allows for the Angles and Reudingians and Auiones to be considered Suebic and Tacitus did while also allowing for North Germanic features to apply as well, instead of breaking them between both North Germanic and Central Germanic.
	
	Additionally, this prevents any particular heritage from being "overloaded"; without this sort of break, East Germanic and Central Germanic are enormous in previous iterations, while now they are only about twice the size of the smallest Germanic heritage.
	
	To become conformal with Vanilla 867, we can have heritage shift events for the Germanic cultures so that Northwestblock shifts to West Germanic, Scandzan to North Germanic, and Suebic to Central Germanic.
	The Anglo/Jute heritage shifts to Central Germanic.
	Przeworsk and Chernyakhov shift to East Germanic.
	If a Germanic culture borders 50\%+1 counties with another Germanic heritage, it'll shift to that heritage instead, modeling in one way what happened to Old Saxon or Jute.
	
	These changes, although extensive, should better reflect the complexities of Germanic cultures during the period and provide a foundation for modeling the eventual harmonization that led to the 3 Vanilla branches evolving.
	Additionally, we could have both Przeworsk and Chernyakhov evolve into an "East Germanic" Heritage, rounding out to a 4 heritage medieval setup if we wanted to support it as an alternate history. 
	
	\subsection{Aesthetics}
	\label{sec:culture_review:subsec:aesthetics}
	A few simple trigger overwrites allows for clothing groups to be more easily defined, instead of admixing them in a way that’s directly seen in the GUI. There’s no new clothing graphics, just new admixtures of existing graphics.
	
	I’ve also worked out how to make custom COA groups (more precisely, make a COA group comprised of other established groups without it throwing errors, so we can make more flexible COA configurations.
	
	Here are the new COA groups and clothing groups:
	
	\subsubsection{Fashion}
	\begin{itemize}
		\item East Germanic: Norse + Northern
		\item Sarmatian: Steppe + Northern
		\item Scythian: Turkic + Steppe
		\item Chernyakhov: Northern + Steppe
		\item Gothic: Northern + Norse + Steppe
	\end{itemize}
	
	\subsubsection{Coat Of Arms}
	\begin{itemize}
		\item Gothic: Norse + Germanic + Baltic + East Slavic
		\item Vistulan: Germanic + Baltic + West Slavic + Celtic
		\item Vandalic: Norse + Germanic + Baltic + West Slavic
		\item Chernyakhov: Germanic + East Slavic + Steppe
		\item Common Slavic: East Slavic + West Slavic + South Slavic
	\end{itemize}
	
	\subsubsection{Ethnicities}
	\begin{itemize}
		\item Vistulan: 50\% West Slavic, 25\% Germanic, 25\% Baltic
		\item Wielbark: Wielbark: 75\% Vistulan, 25\% Norse
		\item Gothic: 50\% Wielbark, 20\% Norse, 20\% East Slavic, 10\% Sarmatian
		\item Sciri: Unique Admixture of Asian, Sarmatian, Dacian, and Germanic
		\item Sarmatian: 50\% Vanilla Saka, 50\% Caucasian
		\item Sakan: 75\% Vanilla Saka, 25\% Caucasian
		\item Xiongnu: 75\% East Asian, 25\% Sakan
	\end{itemize}
	
	\subsection{Culture List}
	Updates from past version are \hl{highlighted}.
	Slashes between traditions indicate replacements if DLC is not available.
	
	\subsubsection{Common Slavic}
	
	Old Slavic:
	\begin{itemize}
		\item The core Slavic culture, sitting at the crossroads of East Germanic, Sarmatian, and Baltic civilization. Going with our idea of making them Chernyakhov, it occupies the forest expanse between the Wild Fields and Polesia.
		\newline
		Revamped Slavic Traditions to give them more flavor and to give them an easier time displacing tribal groups and migrating.
		\item Common Slavic Heritage, Common Slavic Language, Communal Ethos
		\item Forest Folk, \hl{Slavic Traditions}, \hl{Slavic Warfare}
	\end{itemize}
	
	\subsubsection{Baltic}
	
	\paragraph{Dnieper Baltic:}
	This is the mostly unattested branch.
	Some models have it as being a separate branch from East and West Baltic, but we really have no good idea where they fit.
	
	Visutla Veneti:
	\begin{itemize}
		\item The Baltic Veneti we want to have, the same peoples as the core Dneiper Balts of Slavic history. I think because of the large number of unknowns, it should probably speak its own Baltic language rather than group it with East or West Baltic. It gets Wetlanders from its realm, Polygamous to influence later (likely) related Slav groups (Radmichi, \&c.), and Culture Blending to encourage conquerors removing it.
		\item Baltic Heritage, \hl{Venetic Baltic Language}, \hl{Stoic Ethos}
		\item \hl{Wetlanders}, \hl{Polygamous}, Culture Blending
		\item Continental Architecture, Northern Fashion, \hl{Vistulan COA}, \hl{Eastern Equipment}
		\item 50\% Baltic, 25\% East Slavic, 25\% Vistulan
	\end{itemize}
	
	East Galindian:
	\begin{itemize}
		\item The Galindians in the Russian Chronicles. Probably unrelated to the Galindians in the west but for having the same root for their autonym (probably meaning “at the extreme” in proto-Baltic).
		\item Baltic Heritage, \hl{Venetic Baltic Language}, Stoic Ethos
		\item \hl{Wetlanders}, \hl{Polygamous}, Culture Blending
		\item orest Wardens, Hill Dwellers, Staunch Traditionalists
		\item 50\% Baltic, 25\% East Slavic, 25\% Vistulan
	\end{itemize}
	
	\paragraph{East Baltic:}
	Just the Aesti as a proto-culture, in line with their Balto-Finnic neighbors being a proto-culture.
	
	Aesti:
	\begin{itemize}
		\item The Proto-Latvians/Lithuanians.
		\item Baltic Heritage, East Baltic Language, Communal Ethos
		\item \hl{Wetlanders}, \hl{Polygamous}, Culture Blending
	\end{itemize}
	
	\paragraph{West Baltic:}
	Most of these groups were attested as early as the Ist and IInd Century.
	Given some noticeable behavior differences between each other, it is defensible to model them as separate cultures this early on.
	
	(Old) Prussian:
	\begin{itemize}
		\item The Proto-Prussians; oddly not attested early unlike the Sudovians and the (West) Galindians
		\item Baltic Heritage, West Baltic Language, Bureaucratic Ethos
		\item Equitable, Forest Wardens, Sacred Groves
	\end{itemize}
	
	(West) Galindians:
	\begin{itemize}
		\item The Western Galindians; attested to by Ptolemy in the IInd Cenutry.
		\item Baltic Heritage, West Baltic Language, Stoic Ethos
		\item Forest Wardens, Isolationist, Forest Fighters
	\end{itemize}
	
	Sudovians:
	\begin{itemize}
		\item The Sudovians, attested to since Ptolemy. More noted as auxiliaries for the Slavs in the later centuries.
		\item Baltic Heritage, West Baltic Language, Bellicose Ethos
		\item Forest Wardens, Swords for Hire, Xenophilic
	\end{itemize}
	
	\subsubsection{East Germanic}
	This heritage has done a deep scrub.
	
	\paragraph{Oxhöft Group}
	The Oksywie/Oxhöft Group are cultures that formed contemporaneously with the Przeworsk group, from the IInd Century BC to the Ist Century AD.
	They were likely the first casualties of the Gothic Expansion, and should be related-to-but separate from Przeworsk, although notably linked to the West Balts.
	
	Both the Gustow and Rygir are heavily more Germanized than the original Oksywie by the Vth Century. It should still reflect Baltic influence in its COAs and fashion.
	
	Gustow:
	\begin{itemize}
		\item Explicit reference to the Gustow Archaeological Group.
		Probably branched from the Oksywie Culture and was at least partially influenced by both East and Central Germanic cultures.
		Presuming the Rugii were the elite portion expelled by the proto-Goths when they landed on the Baltic coast and formed the early Wielbark culture.
		By the Vth Century, represents the tail end of the Oskywie peoples who merged with Elbe groups, which is why it speaks Elbe Germanic.
		
		By the Vth Century, heavily Germanized compared to the original Oksywie.
		\begin{itemize}
			\item \hl{Custom localization explaining this}
		\end{itemize}
		\item East Germanic Heritage, Elbe Germanic Language, Stoic Ethos
		\item Forest Folk, \hl{Isolationist}
		\item Continental Architecture,\hl{Vistulan COA, Northern Clothing, Northern Equipment}
		\item 75\% Central Germanic, 25\% Vistulan
	\end{itemize}
	
	Rygir:
	\begin{itemize}
		\item Referenced as early as the Ist Century by Tacitus, and Jordanes mentions the Goths expelling the Ulmerugi, so they are probably the mobile, elite portion. Similarly Germanized as an effect of their wandering in Germanic territories.
		\item East Germanic Heritage, \hl{Vandalic Language}, Stoic Ethos
		\item Concubies, Isolationist, Warrior Culture
		\item Continental Architecture,\hl{Vistulan COA, Northern Clothing, Northern Equipment}
		\item 75\% Central Germanic, 25\% Vistulan
	\end{itemize}
	
	Skirian:
	\begin{itemize}
		\item The Sciri remain a constant mystery. They are very early attested but pop in and out of history, starting in the IInd Century BC with some raids, but they only are really clearly attested around 300 AD. 
		
		Given their links to the Bastarnae, I suspect the “Pure/Bastard” distinction might hold weight, with the Sciri being an unmixed people (at least initially). On the other hand, Odovacar’s father being a Hun undercuts that a bit.
		If we are calling the Heruli the inheritors of the Bastarnae, then maybe this still works out well.
		
		I personally suspect they were some otherwise less mixed group that readily admixed under pressure and competition with the better attested Bastarnae.
		Since they would be in the same area as the Chernyakhov, I see no reason not to let them be Gothic speakers (or at least, whatever East Germanic they spoke became Gothic enough for game mechanics during the Chernyakhov times).
		
		So with that in mind, I think making them essentially the original East Germans of the Pomeranian Culture [along with the Bastarnae] (Pommeranian culture itself was likely a successor of the non-Germanic Lusatian culture); some proposals for the evolution of Germanic languages have it in the mouth of the Vistula ca. 500 BC, which would align roughly with the decline of the Lusatian culture and the rise of Pommeranian.
		Further Germanization probably occurred in the Chernyakhov period.
		
		Their innate conservativeness means they’ll use Northern Equipment instead of Norse or Continental
		\item \hl{Not descended from any other culture, notes on its strangeness}
		\item East Germanic Heritage, Skirian Language, Bellicose Ethos
		\item Warrior Culture, Isolationist, Ruling Caste, Stand and Fight!, Concubines
		\item \hl{Continental Architecture, Vistulan COA, Northern Clothing, Northern Equipment}
		\item 100\% Skirian
	\end{itemize}
	
	\paragraph{Przeworsk Group}
	This culture spanned from Vth Century BC to the Vth Century AD and was tied to the Vandals and (possibly) the Burgundians.
	Przeworsk had some clear Germanic features but also inputs from Baltic and Slavic groups, with a Genetic link to later West Slavs \cite{SlavGenomes}.
	So as the Vandals and Burgundians represent elite groups that left/were forced to leave, while Przeworsk (by the mid-IVth Century) more represents the peasant culture.
	
	So, ethnically, Przeworsk is a strong mix of groups, while Vandalic and Burgundians are much more Germanic (and primarily Germanic instead of Norse, with Jastorf being the contributor initially).
	
	Przeworsk:
	\begin{itemize}
		\item The Proto-Vandalic culture, probably significantly Germanized by the Vandal period. On the downswing, with the Langobards muscling in to the west and Slavs muscling in from the east.
		It should represent more of the multi-ethnic “peasant” component of the old Przeworsk structure, as per Heather’s argument and be the part that has more Celtic and Baltic influences.
		\item East Germanic Heritage, \hl{Vandalic Language}, Cosmopolitan Ethos
		\item Forest Folk, \hl{Collective Lands}, Culture Blending/Xenophilic
		\item \hl{Continental Architecture, Vistulan COA, East Germanic Clothing, Northern Equipment}
		\item 100\% Vistulan
	\end{itemize}
	
	Vandalic:
	\begin{itemize}
		\item The very Germanic ruling component (focusing on the Hasdingi component here) of the Przeworsk, that split off circa the middle of the IVth Century.
		Aesthetics should be Germanic over anything else, but with some hints of Celtic/Baltic/Dacian influence.
		\item East Germanic Heritage, Vandalic Language, Bellicose Ethos
		\item Strong Believers, Isolationist, Concubines, Ruling Caste, Vandalic Warfare
		\item \hl{Norse Architecture, Vandalic COA, East Germanic Clothing, Norse Equipment}
		\item 50\% Vistulan, 25\% Norse, 25\% Germanic
	\end{itemize}
	
	Burgundian:
	\begin{itemize}
		\item Utilizing \textit{Getica}, the author thinks the Burgundian migration was likely precipitated by the expansion of the Wielbark Group into former Przeworsk lands, which would make them relations of the Vandals.
		Not direct descent, but relations.
		Same Aesthetics as Vandalic (being a likely elite component), but not Norse, being driven out by the Norse.
		\item Divergence from Przeworsk with notes
		\item East Germanic Heritage, \hl{Vandalic Language}, Cosmopolitan Ethos
		\item Religion Blending, Concubines, Ruling Caste, Culture Blending/Xenophilic
		\item \hl{Continental Architecture, Vistulan COA, HRE/Western Clothing, Continental Equipment}
		\item 50\% Vistulan, 50\% Germanic
	\end{itemize}
	
	\paragraph{Wielbark Group}
	This group is associated with the Goths and their arrival, admixed somewhat with the existing Przeworsk groups.
	Expanded at the expense of the old Prezeworsk presence in the Vistula basin, likely contributing to the eventual Chernyakhov culture.
	
	Wielbark:
	\begin{itemize}
		\item This is the Germanic Wielbark culture which contributed greatly to the Gothic ethnogenesis.
		It has a number of shared hallmarks with the Norse, especially stone circle raising and stelae.
		I think Heather’s elite migration hypothesis is likely correct, so this would represent the mostly peasant remainder (so, having some Baltic and West Slavic ethnicity/features).
		Same Aesthetics as Vandalic (being a likely elite component), but not Norse, being driven out by the Norse.
		\item Custom Localization
		\item \hl{East Germanic Heritage, Gothic Language, Communal Ethos}
		\item \hl{Northern Stories/Runestone Raisers, Collective Lands, Forest Folk}
		\item \hl{Continental Architecture, Wielbark COA, East Germanic Fashion, Continental Equipment}
		\item 100\% Wielbark
	\end{itemize}
	
	Denziner:
	\begin{itemize}
		\item A breaking off of the Wielbark culture along the coast.
		More Northern European than the preceding Wielbark, with influences from Elbe Germans and Gustow as well.
		It gets the ethos from Gustow, the language and “Sacred Groves” from Lebus.
		\item \hl{Wielbark/Lebus hybrid}
		\item East Germanic Heritage, Elbe Germanic Language, \hl{Stoic Ethos}
		\item \hl{Northern Stories/Runestone Raisers, Forest Folk, Sacred Groves}
		\item \hl{Continental Architecture, Wielbark COA, East Germanic Fashion, Continental Equipment}
		\item 75\% Wielbark, 25\% Norse
	\end{itemize}
	
	Vidivarri:
	\begin{itemize}
		\item Likely a Baltic backscatter, probably the pre-Slavic force that wiped out the last of the Wielbark ca. early VIth Century \cite{Vidivarii}
		\item \hl{Wielbark/Prussian Hybrid early late 400s/early 500s (i.e., near 476, and after the collapse of the Hunnic Empire on Anastasius’ horde and timeline Heather provides)} \cite{Vidivarii,HeatherEmpiresAndBarbarians}
		\item East Germanic Heritage, \hl{Gothic Language, Bureaucratic Ethos}
		\item Forest Folk, Sacred Groves
		\item \hl{Continental Architecture, Baltic COA, Baltic Fashion, Northern Equipment}
		\item 50\% Wielbark, 50\% Baltic
	\end{itemize}
	
	Gepids:
	\begin{itemize}
		\item The Gepids remain relatively mysterious, but we have two primary anchors:  the Gepids are referenced in Getica as a Gothic people, but they are only definitively linked to the area of the Tisza or Koros rivers. If Jordanes is accurate in his descriptions of the Gepids as being from “rugged mountains”, they probably originated from the upper Dniester lands.
		
		Genetics points to a Wielbark and Langobard \cite{GepidGenetics}, but the grave is from the VIth century. So probably Wielbark inheritance, at least for the elite.
		
		Given Jordanes listing them as Goths and their proximity to the core Gothic civilization, I’m inclined to make them Gothic speakers, but not directly related to Goths (sister cultures), considering their mutual hostility. The Gepids probably moved into the vaccuum left by the Vandals and amalgamated whatever Germanics remained (probably relic Victohali and Agaragantes).
		
		Get Recognition of Talent to replace Loyal Subjects (seems more fitting, since they turned on the Huns).
		\item \hl{Wielbark/Divergence with Sarmatian and Dacian elements}
		\item East Germanic Heritage, \hl{Gothic Language}, Bellicose Ethos
		\item Stand and Fight!, \hl{Strength in Numbers, Recognition of Talent}
		\item Continental Architecture, \hl{Wielbark COA, East Germanic Fashion}, Northern Equipment
		\item 50\% Wielbark, 25\% Sarmatian, 25\% Dacian
	\end{itemize}
	
	\paragraph{Chernyakhov Group}
	This group is associated with the Chernhyakhov Complex, which was almost certainly an amalgamation of Germanic, Slavic, Sarmatian, and Dacian influences.
	The most common arguments seem to be it was at least Gothic-dominated \cite{HeatherEmpiresAndBarbarians}, as it arises right before the Goths migrate into the Roman Empire.
	
	Heruli:
	\begin{itemize}
		\item The Heruli appear more-or-less contemporaneously with the Goths in the historical record, possibly as a branch of the Germanics living near the Pontic Steppe in the late IVth Century.
		Beyond that, we don’t have much.
		The large presence of non-Germanic names \cite{OdoacerGermanOrHun} makes the author think it was either a Germanic elite with large non-Germanic components or an admixed Germanic group.
		For flavor, I’m planning the later (maybe the remnant Bastarnae?), so it’ll have a mix of Germanic and Dacian/Sarmatian aesthetics.
		
		Communal Ethos chosen to be a more proper Sarmatian/Gothic Hybrid per game mechanics
		\item \hl{Gothic/Sarmatian hybrid like Taifals, but more Gothic than Sarmatian; custom loc}
		\item East Germanic Heritage, Gothic Language, \hl{Communal Ethos}
		\item Swords for Hire, Gothic Warfare, \hl{Warrior Culture}
		\item Continental Architecture, \hl{Chernyakhov COA, Chernyakhov Fashion}, Norse Equipment
		\item 60\% Wielbark, 20\% Sarmatian, 10\% Dacian, 10\% Hun
	\end{itemize}
	
	Taifals:
	\begin{itemize}
		\item The other mysterious Gothic/Sarmatian Hybrid Culture, this being more Sarmatian than Gothic, being gifted cavalry.
		Could also be argued to be a Sarmatian culture, but given it's clearly Chernyakhov, it should be of the same heritage as the rest of the Chernyakhov cultures.
		\item \hl{Sarmatian/Gothic Hybrid}
		\item East Germanic Heritage, Scythian Language, Bellicose Ethos
		\item Martial Admiration, Concubines, Swords for Hire, Horse Lords, Staunch Traditionalists
		\item Eurasian Steppe Architecture, \hl{Chernyakhov COA, Chernyakhov Fashion}, Mongolian Equipment
		\item 50\% Sarmatian, 30\% Wielbark, 10\% Dacian, 10\% Hun
	\end{itemize}
	
	Gothic:
	\begin{itemize}
		\item The progenitor culture of all the explicit Gothic cultures.
		Almost certainly hybridized with the existing Germanic groups (Skirian, Bastarnae) and achieved dominance over the region at least west of the Dneiper prior to the Hunnic arrival.
		Was probably an elite component \cite{HeatherEmpiresAndBarbarians}, so should be more Norse than Wielbark.
		
		Changing to a more Bellicose Ethos seems to match Gothic dominating behavior better and serve as a launch pad for better consideration of the Gothic Sub-Cultures Ethos.
		\item \hl{Divergence from Wielbark with extensive notes}
		\item East Germanic Heritage, Gothic Language, \hl{Bellicose Ethos}
		\item Martial Admiration, Gothic Warfare, \hl{Ruling Caste}
		\item Norse Architecture, \hl{Gothic COA, Gothic Fashion}, Norse Equipment
		\item 100\% Gothic
	\end{itemize}
	
	Visigothic:
	\begin{itemize}
		\item The Thervingi; likely a tribal designation that evolved into an ethnic one circa 410s.
		
		The Visigoths, post 410s, were not exactly known for their victorious war-fighting (given they regularly lost land throughout the period).
		The author thinks Communal might actually fit better; the Visigoths were one of the few Post-Roman states to build new holdings, they remained insular compared to their Hispano-Roman population until near the end, and a mercenary buff is not really warranted in their case.
		\item Divergence from Gothic in 410s, with notes
		\item East Germanic Heritage, Gothic Language, \hl{Communal Ethos}
		\item Martial Admiration, Visigothic Codes, Concubines, Ruling Caste, Gothic Warfare
		\item Norse Architecture, \hl{Gothic COA, Gothic Fashion}, Norse Equipment
		\item 100\% Gothic
	\end{itemize}
	
	Ostrogothic:
	\begin{itemize}
		\item The main Greuthungi branch; likely a tribal designation, eventually cast off to be the “Eastern” Goths when the Visigoths became a polity.
		
		Using Stoic Ethos to be what it diverges from Gothic with for Game mechanics and the mercenary buff seems unwarranted like the Visigoths
		\item Formed in 454 after Nedao
		\item East Germanic Heritage, Gothic Language, \hl{Stoic Ethos}
		\item Martial Admiration, Concubines, Gothic Warfare, Quarrelsome, \hl{Ruling Caste}
		\item Norse Architecture, \hl{Gothic COA, Gothic Fashion}, Norse Equipment
		\item 100\% Gothic
	\end{itemize}
	
	Thracian Gothic:
	\begin{itemize}
		\item The secondary Greuthungi branch; likely a tribal designation, eventually cast off to be the “Eastern” Goths when the Visigoths became a polity.
		
		This would change to arise after Theodoric took control of the Moseogoths and took the elite component away.
		\item \hl{Formed in early Vth Century (minimum time period for divergence from Ostogothic in game mechanics terms)}
		\item East Germanic Heritage, Gothic Language, Communal Ethos
		\item Strong Believers, Concubines, Pastoralists (No Tribal Unity)
		\item Norse Architecture, \hl{Gothic COA, Gothic Fashion}, Norse Equipment
		\item 100\% Gothic
	\end{itemize}
	
	Crimean Goths:
	\begin{itemize}
		\item The only Goths to survive until early modern times, well isolated on Crimea
		\item Formed in 376 after Hunnic Invasion
		\item East Germanic Heritage, Gothic Language, \hl{Stoic Ethos}
		\item Loyal Subjects, Gothic Warfare, Staunch Traditionalists
		\item Norse Architecture, \hl{Gothic COA, Gothic Fashion}, Norse Equipment
		\item 100\% Gothic
	\end{itemize}
	
	\subsubsection{Central Germanic}
	Lebus:
	\begin{itemize}
		\item No documentation; the author guesses the Liebesitz culture?
		Next to nothing on this, so inheritors of the Semnones?
		It’s what I’m basing the Traditions and Ethos on.
		\item Central Germanic Heritage, Elbe Germanic Language, Spiritual Ethos
		\item Sacred Groves, Forest Folk, Forest Fighters
	\end{itemize}
	
	Warnic:
	\begin{itemize}
		\item Migrational Era Warini; supposedly a bit of threat to the Franks ca. Vth Century
		\item Central Germanic Heritage, Elbe Germanic Language, Bellicose Ethos
		\item \hl{Warrior Culture, Stand and Fight!}
	\end{itemize}
	
	Langobard:
	\begin{itemize}
		\item Migrational Era Langobards.
		Played a role in the collapse of the Gepids.
		Migrational Path is somewhat undetermined, but was within Suebic lands for a long time.
		\item Central Germanic Heritage, Elbe Germanic Language, Bellicose Ethos
		\item \hl{Warrior Culture, Stand and Fight!}
	\end{itemize}
	
	Angles, Buri, Reudingian
	
	\subsection{West Germanic}
	Saxons:
	\begin{itemize}
		\item Migrational Era Saxons.
		Arose around IIIrd or IVth Century AD.
		More associated with the Franks than the Suebians, so should be West Germanic instead of Central Germanic.
		\item \textit{West Germanic Heritage}, North Sea Germanic Language, Bellicose Ethos
		\item Practiced Pirates, Ting-Meet, \hl{Coastal Warriors/Hirds}
	\end{itemize}
	
	\subsubsection{North Germanic}
	Geats, Swedes, Norse, Danes, Gutnish, Auiones
	
	\subsubsection{Proto-Carpathian}
	Dacian:
	\begin{itemize}
		\item Really more of the tribal Dacians which escaped Roman dominance.
		Shifted to Mountain Homes to better fit their aspect as a relic population
		\item Proto-Carpathian Heritage, Dacian Language, Spiritual Ethos
		\item Philosopher Culture, Unblemished Rulership, Adaptive Skirmishing, \hl{Mountain Homes}
	\end{itemize}
	
	\subsubsection{Balto-Finns}
	Fenni:
	\begin{itemize}
		\item The split of Balto-Finnic between Saami and the rest occurred well before the Mod began, but the split into Vepsian, Estonian, Finnish, and Karelian occurred during or shortly after the mod period.
		Since it seems to be a common theory, the Proto-Balto-Finns will be called Fenni as per Tacitus
		\item Balto-Finnic Heritage, Finnic Language, Communal Ethos
		\item Forest Wardens, Forest Folk
	\end{itemize}
	
	\newpage
	
	\subsection{Tradition Review}
	\label{sec:culture_review:subsec:tradition_review}
	
	\section{Religion Review}
	\label{sec:religion_review}
	
	\subsection{Turks}
	\label{sec:religion_review:subsec:turks}
	Bolghars? Monotheistic/Henotheistic Tengrism? How would it be different than Mongolic Tengrism?
	
	\newpage
	
	\section{Bibliography}
	\label{sec:Bibliography}
	
	\begin{thebibliography}{99}
		%Key Sources
		\bibitem{PLRE_Vol2}
		Jones, A.H.M., Martindale, J.R., Morris J. \textit{The Prosopography of the Later Roman Empire, Vol 2}. University Press. 1971.
		\bibitem{HeatherEmpiresAndBarbarians}
		Heather, P. “Empire and Barbarians: The Fall of Rome and the Birth of Europe.”
		\bibitem{IndoEuroEncyclopedia}
		Mallory, J.P. and Adams, D.Q. "Encyclopedia of Indo-European Culture". Fitzroy Dearborn Publishers. 1997.
		%Germans
		\bibitem{WolframHistoryOfTheGoths}
		Wolfram, H. \textit{History of the Goths}. University of California Press. 1990
		\bibitem{PrzeworskHistory}
		\textit{The Past Societies: Polish Lands from the First Evidence of Human Presence to the Early Middle Ages}. "Chapter 6: Przeworsk culture, society, and its long-distance contacts, AD 1-250". Institute of Archaeology and Ethnology, Polish Academy of Sciences.
		\bibitem{Vidivarii}
		Kontny, B. “How did the Vidivarii emerge? The transition between Germanic and Balt settlement in the late 5th c. in northern Poland.” MPOV – Beyond the Odra and the Vistula.
		\bibitem{GepidGenetics}
		Ginguta, A. “Maternal Lineages of Gepids from Transylvania”. <https://www.academia.edu/76197565/Maternal\%20Lineages\%20of\%20Gepids\%20from\%20Transylvania>
		\bibitem{OdoacerGermanOrHun}
		Reynolds, R. L. and Lopez, R. S. “Odoacer: German or Hun?” <https://www.jstor.org/stable/1845067>
		\bibitem{DieSachsens}
		Springer, M. \textit{Die Sachsen}. Kohlhammer Verlag. Sept 16, 2004. pg. 24 (German)
		\newline\specificCite{Die Saxones werden hier im selben Atemzug mit den Franken als Bundesgenossen des Magnentius gennant...}.
		\bibitem{ToddEarlyGermans}
		Todd, M. \textit{The Early Germans}. Wiley \& Sons, 2009.
		\bibitem{GepidsInBalkans}
		Kharalambieva, Anna (2010). "Gepids in the Balkans: A Survey of the Archaeological Evidence". In Curta, Florin (ed.). Neglected Barbarians. Studies in the early Middle Ages, volume 32 (second ed.). Turnhout, Belgium: Brepols. pp. 245–262. 
		%Huns
		\bibitem{OttoHuns}
		Maenchen-Helfen, O. J., Knight M. (Ed). \textit{The World of the Huns: Studies in their History and Culture}. University of California Press. 1973.
		\bibitem{CambridgeHistoryEarlyInnerAsia}
		Sinor, D. \textit{The Cambridge History of Early Inner Asia}. Cambridge University Press. 1990.
		\bibitem{KimHuns}
		Kim, H.J. \textit{The Huns}. Routledge, New York. 2016.
		%Alans
		\bibitem{BachrachAlans}
		Bachrach, B. S. \textit{A History of the Alans in the West}. University of Minnesota Press. 1973.
		%Balts
		\bibitem{BalticHydronyms}
		Baltic Hydronyms map.
		%Finns & Saami
		\bibitem{LaplandSaami}
		Akiko, Ante. "How did Lapland become Saami? Reconstructing the interaction of Proto-Saami, Proto-Norse and Palaeo-Laplandic language communities in the Iron Age". Presentation at "Contacts: Archaeology, genetics and languages"
		\bibitem{DiversificationOfProtoFinnic}
		Kallio, Petri. "The Diversification of Proto-Finnic" Fibula, Fabula, Fact: The Viking Age in Finland, pp. 155-168. Studia Fennica Historica 18. Helsinki. 2014.
		\bibitem{SaamiMap}
		Boradbent, N. D. "Lapps and Labyrinths:  Saami Prehistory, Colonization and Cultural Resilience." pg. 42. Smithsonian. 2010. \specificCite{Figure 29 provides a map of the probable Saami prehistorical settlement.}		
		%Slavs
		\bibitem{EmergenceOfRussia}
		Franklin, Simon; Shepard, Jonathan (2014-06-06). The Emergence of Russia 750–1200. Routledge. p. 101. ISBN 978-1-317-87224-5.
		\bibitem{SlavGenomes}
		Mielnik-Sikorska, M. et al. “The History of the Slavs Inferred from Complete Mitochondrial Genome Sequences.” <https://journals.plos.org/plosone/article?id=10.1371/journal.pone.0054360>
	\end{thebibliography}
	
	\newpage
	
	\section{Appendix I: New Ethnicities}
	\label{sec:appendix_new_ethnicities}
	
\end{document}