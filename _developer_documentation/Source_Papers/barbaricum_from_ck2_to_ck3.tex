\documentclass{article}
\usepackage{fullpage}
\usepackage{graphicx}
\usepackage{geometry}
\usepackage{xcolor,soul}

\sethlcolor{yellow}

\graphicspath{./images}

\title{Barbaricum: Updating the Map from CK2 to CK3}

\begin{document}
	\newgeometry{left=2cm,bottom=2cm,right=2cm,top=2cm}
	
	\maketitle
	
	\section{Introduction}
	\label{sec:intro}
	
	The origin of the Balts/Slavs/Goths are a contentious issue, and the CK2 culture set up has some noticeable issues, not the least of which is very early Slavic migration to Novgord, which seems to clash with a later settlement of Northern Russia by Slavs in literature \cite{EmergenceOfRussia}.
	
	Before getting into the meat of the issue, a few notes:
	
	\begin{itemize}
		\item It is doubtful that many archaeological complexes were tied to a single culture (in CK3 terms) and it is complicated by cultures borrowing materially from each other.
		\item Actual cultures (in CK3 terms) probably spanned the borders between Archaeological Complexes.
		\item Actual ethnogenesis is a complicated process, and the hybridization/divergence system in CK3, while a huge step forward from CK2, is a gross oversimplification.
		\item We want a long-standing Baltic Veneti culture interfacing with Przeworsk, which was the model in CK2. This would also make the Neuri of Heroditus’ time Balts, which seems to be a consensus.
		\item The Urheimat of the Slavs should on the border of Baltic and Sarmatian zones, mostly to identify the “Scythian Farmers”.
		Florin Curta’s argument in favor of a Dacian-region is a minority position; although he has a point about ethnicity being a modern external construct being imposed backwards, it fails to cover well the expansion of Slavic languages, which is the central concern in this article.
		In which case the Slavic Urheimat should be closer to the generally agreed center.
		\item LT-Rascek is of the opinion that the Slavs had a later ethnogensis, and for linguistic reasons, it had to initial occur alongside the Balts.
		The map (Map 17 of \cite{HeatherEmpiresAndBarbarians}) of all the proposed Slavic Urheimat has only a few that would fit, namely G: “Baran/Godlowski – Carpathian”, C: “Rusanova – Polesia”, and “D: Tretiakov – Kievan”.
		Considering the idea that the Slavs were essentially a border people taking influences from Przeworsk and Chernyakhov, that narrows the region down to the northern Chernyakhov complex, with the Baltic Veneti occupying Polesia
		\begin{itemize}
			\item LT-Rascek thinks the references to the Dnieper Balts strengthens the later migration argument.
			The Kolochin \& Moshchiny cultures were probably transitional as Slavicization occurred and the timeline matches pretty well. So they should either be divergences from Slavic or hybrid Slavic/Baltic cultures.
		\end{itemize}
		\item Heather's "elite migration" arguments are probably a good model for what happened with the Goths and Vandals and explain the decline in material goods in the Wielbark and Przeworsk zones, especially considering the best raiding target (Rome) was historically to the south.
		Much like how nomadic elites were built on pillage and trade with settled groups, if we think that Oium represented a Gothic dominated pillage/trade confederation, it would make sense that the Przeworsk and Wielbark zones would become denuded of rulers in time.
		With some tradition modifications, we can make these cultures model more peasant/low holding and easier to displace by the Slavs.
		\item CK3 added functional landless gameplay, which unlocks some options for representing certain aspects of history, such as roving warbands or nomadic groups that we can more readily split the political and cultural/religious maps.
		So these maps will focus primarily on the culture/religion changes while leaving the political map for a different document.
		\item \textit{WtWSMS} should really consolidate research into something indexable, like papers or articles
	\end{itemize}
	
	This article is divided into the following sections:
	
	\begin{itemize}
		\item Section \ref{sec:high_level_overview}, a high level overview providing a more narrative approach to the problem
		\item Section \ref{sec:timeline}, a more period-by-period view of the evolution of the culture map
		\item Section \ref{sec:culture_review}, reviewing changes to individual cultures
		\item Section \ref{sec:Bibliography}, for sources for future review
		\item Section \ref{sec:appendix_new_ethnicities}, for a detailed breakdown of ethnicity math
	\end{itemize}
	
	\section{High Level Overview}
	\label{sec:high_level_overview}
	
	\subsection{General Archaeological Complex Relations}
	In this diagram, the author tries and links a number of Archaeological complexes in terms of both influence (thick lines) and succession \& contribution (thin lines).
	The only unambiguous Germanic complex is the Jastorf,  the only unambiguous Slavic complex is the Kiev complex, and the Sarmatians are unambiguously Sarmatian.
	All the rest are debated, but the author think marking the Brushed Pottery culture as Baltic and the La Tene as Celtic is defensible and traditional.
	
	\begin{figure}[h]
		\centering
		\includegraphics[width=0.75\textwidth]{./images/archeological_complex_relations.png}
		\caption{Archeological Complex Relations.\newline \small \textit{Relations between various archaeological complexes.
				The only truly unambiguous cultures are Germanic Jastorf, Baltic Brushed Pottery, Sarmatian, and Slavic Kiev Culture.
				The remainder all show some degree of cultural admixture.}}
	\end{figure}
	
	\subsection{General Narrative}
	It remains difficult to know precisely what the situation in Barbaricum was before the Middle Ages, we can at least build a general narrative for self-consistency.
	The author presume there was an initial phase of Germanization prior to the Goths arrival and that \textit{Getica} is at least partly historical; there’s good genetic evidence that the Goths (or at least the elites) were originally from Scandinavia and their migration to the Vistula precipitated a number of changes among the Przeworsk, as well as likely the Zarubintsy – possibly breaking the continuum between the two to allow the Zarubintsy to fully Slavicize.
	
	It is likely the Lusatian culture was Baltic, or at least non-Germanic.
	Most models for the spread of the Germanic languages have the mouth of the Vistula Germanizing between 500 BC and nearly the whole Vistula Germanizing by Augustus’ time \cite{HeatherEmpiresAndBarbarians}.
	This would make the Pomeranian Culture, the antecedent of Przeworsk and Oksywie Cultures, the “Proto-East Germanic” Culture, although with significant Baltic/non-Germanic influences.
	The later two were probably fully “German”, although the author suspects that the Jastorf played a bigger role than Norse in this period.
	As for the Balts, the author presumes their original extent was at least the generally agreed Baltic Hydronym region \cite{HeatherEmpiresAndBarbarians,BalticHydronyms}.
	
	The Przeworsk would be an admixture and still borrow and exchange with the Zarubintsy, which tend to be viewed as either a sister group of Przeworsk or its eastern extension. So around 200 BC it’d probably look something like this
	
	\begin{figure}[h!]
		\centering
		\includegraphics[width=1.00\textwidth]{./images/archeological_complexes_200BC.png}
		\caption{Archaeological Complexes ca 200 BC.
		\tiny Zarubintsy extents taken from \cite{IndoEuroEncyclopedia}, Przeworsk extents taken from \cite{PrzeworskHistory} (Early Phase), Baltic Extents (including Zarubintsy) taken from \cite{BalticHydronyms}, Saami extents taken from \cite{LaplandSaami}, and Balto-Finnic is extrapolated.
		Note that Przeworsk is abutting the La Tene Zone in modern Bohemia and Jastorff in Germania, serving as a meeting point between Balto-Slavic, Celtic, and Germanic influences.}
	\end{figure}
	
	Around the IIIrd Century BC, Pomeranian gave way to Oksywie and Przeworsk.
	The former we don’t have as much data, as it is not a culture in CK2 WtWSMS, but being right along the Baltic coast, it was the first to receive the settlers/colonizers from Scandinavia.
	For this reason and the otherwise limited expanse of Germanic during the period, the author associatse them with the sudden appearance of the generally agreed Germanic Bastarnae and Skirians at the Greco-Roman border ca. 200 BC.
	So the original Skirian appearance in the South probably portended the migrations that would follow, and the Skirians and Bastarnae were probably semi-Germanic early migrators.
	
	The author suspects that this nucleus would remain near the Vistula mouth and, besides pushing the Skirians and Bastarnae to the Carpathians, primarily be the final Germanizing of Oksywie and Przeworsk.
	That would remain the process for the next few hundred years.
	Then, the Wielbark culture arrives in the Ist Century AD.
	
	As Heather points out \cite{HeatherEmpiresAndBarbarians}, Wielbark expanded at the expense of Przeworsk, down the Vistula river and this lines up with \textit{Getica} saying the Goths arrived from Scandinavia and eventually end up in Oium (generally agreed to be modern Ukraine, at least partially).
	Under this narrative, Wielbark represents the “original” Gothic culture, or more precisely, its form prior to the Chernyakhov period where Goths dominated what is now Moldova and (at least) Ukraine west of the Dneiper.
	Over the following centuries, Wielbark would expand down the Vistula, partly breaking the continuity between Zarubnitsy and Przeworsk and likely facilitated the further Germanization of Przeworsk. 
	That said, Przeworsk was also under change during the period; by the late Roman Period, part of its material culture arose in the headwaters of the Tisza \cite{PrzeworskHistory}. This is likely related to the proto-Vandalic groups, which seem to be attested to in the region circa AD 270s.
	
	Wielbark would continue extending down the Vistula for the next few hundred years in a sort of transient way \cite{HeatherEmpiresAndBarbarians}, where it then came in contact with the Sarmatians and/or Slavs.
	There, it became a new Chernyakhov culture from the IInd to the Vth centuries.
	The author suspects Chernyakhov were truly multi-ethnic with a dominant Germanic (namely Gothic) component.
	The author further suspects there’s an element of truth to Oium in \textit{Getica}, although it is probably a little less of an empire and more a broad, confederal Gothic dominance of the early Slavs, Dacians, and Sarmatians, who intermingled in a way that creates a common shared material culture.
	This would last until the late IVth and early Vth Centuries, when the Huns attack, sending the Goths fleeing into the Roman Empire and otherwise generally shuffling around peoples and polities of eastern barbaria. 
	
	\begin{figure}[h!]
		\centering
		\includegraphics[width=1.00\textwidth]{./images/archeological_complexes_350AD.png}
		\caption{Archaeological Complexes ca. AD 350.
		\newline\tiny Wielbark has expanded down to modern Ukraine, giving birth to the Chernkyakhov Archeological Complex.
		Przeworsk has found itself somewhat marginalized, but components have moved into the chaotic lands of former Dacia; this is either Gepid or Vandalic, probably Gepid is \textit{Getica} is referencing some actual conflicts. Known Iazyges lands provided for comparison. Chernyakov extents taken from \cite{HeatherEmpiresAndBarbarians,IndoEuroEncyclopedia}, Kyiv extents taken from \cite{IndoEuroEncyclopedia} under Early Kolochin}
	\end{figure}
	
	The Germanic Groups that fled into the Empire, namely the Goths and Vandals, were probably an elite, aristocratic component and their fleeing is what precipitated the collapse of Germanic Wielbark and Prezworsk in the late IVth and early Vth centuries in Ukraine and Poland \cite{HeatherEmpiresAndBarbarians}.
	Other groups, like the Slavs, were probably subordinate or simply poorer (Gepids, Heruli, Skirians) and the Huns would make use of them extensively as soldiers because they were able to turn on their former masters/wealthier counterparts or were, like the Goths and Vandals, pushed away from the general Chernyakhov region in favor of the new Huns.
	This would mirror the collapse of the Rouran in the East and the rise of the Gokturks centuries later.
	
	The author strongly suspects the Slavs filled the vacuum in Ukraine and Southern Poland left by the Germans fleeing as this region overlaps well with the better attestation of Attila’s Hunnic realm.
	The relatively flat social structure of the Slavs during the time also probably played a role in the Slaviziation of the region, given the Germanic elite, skilled component ended up fleeing towards the empire (which Heather hypothesizes \cite{HeatherEmpiresAndBarbarians}).
	This dovetails with CK2 WtWSMS’s sudden appearance of Lechitic culture in modern Poland around 525 and they seem to be the most likely group to replace the Germans there, if we go by the Slavs living in the woods between the Baltic and Sarmatian peoples (which plays well with keep Balto-Slavic relatively adjoined until late in their history and the notion that the proto-Slavs are the "Scythian Farmers" of antiquity), before reaching their final form in the Kyiv/Kolochin groups.
	
	So, by the early Vth Century, the Huns have utterly reorganized the landscape culturally and groups have moved around significantly, as was what regularly happened in what is now the Mongolic and Turkic Steppe.
	Damage to the Przeworsk and Wielbark and the displacement of the Skirians and Heruli from Moldova and the Gepids from Slovakia opened up significant lands to Slavic settlement, entirely remaking the map.
	
	From here, the Slavs would move up the Vistula, especially when the Lombards flee, taking the last of the well-organized Germanic able to counteract Slavization, and the collapse of the Huns in the Vth Century creates room on the lower Danube for the proto-Sklaveni and proto-Antes to settle, where the Byzantines historically first met them.
	So by the early-to-middle VIth Century, Germanic cultures on the Vistula are gone, which coincides with the rise of new material cultures in the VIth and VIIth centuries that are unambiguously Slavic, especially in regions where the Eastern Germanics once lived.
	
	So around the 550s, the proto-West and proto-South Slavs are in position to spread, while the Dneiper Balts remain in time for the proto-East Slavs to push north and steadily displace them in a process that wouldn’t be completed until the early Middle Ages \cite{EmergenceOfRussia} and would more readily justify oral history recording their existence (compared to the seeming lack of oral history where the West Slavs displaced the East Germanics).
	
	The Avars arrive crica 560 and eventually bring the Slavs under their domination, helping fuel their push to the south and west that would complete the Slavization of much of the Balkans by the proto-South Slavs, push the proto-West Slavs are far Elbe, and push the East Slavs to the North, displacing the Balts and Balto-Finns in their drive north.
	By 600, I would suspect the culture map would start reflecting the position in Vanilla CK3.
	The Vlach remain an open and contentious question, but the CK3 systems really don't model peripatetic peoples living contemporaneously with settled populations; given Vanilla CK3 gives them pastoralists, they probably were an amalgamation of Romance speakers with nomads during the Pannonian Avar period.
	
	Beyond the reaches of the Balts, the situation in Scandinavia is also problematic on the CK2 map. Actual arrival of the Finns in Finland is dated to at the earliest, when WtWSMS \cite{LaplandSaami} (See \cite{SaamiMap} for a good breakdown of extents of Saami prehistorical settlement in Norway/Sweden).
	Before then, there was a likely Paleo-Laplandic speaking peoples in the region, who were eventually subsumed into the Saami; this, however, occurred well before the start of the mod (or at most, the Paleo-Laplandics died out before the Viking Era). 
	For simplicity, the author proposes we simply lump the Paleo-Laplandics in with the Saami and not represent their possible surivial into the mod period.
	
	Regardless, the Balto Finns remained to the South of Finland until around the VIth century, although the development of the Balto-Finnic civilization was underway, with linguistic splits into Estonian, Finnish, and Karelian occurring between 600 Bc and AD 150 and only later splitting off \cite{LaplandSaami,DiversificationOfProtoFinnic}.
	\cite{DiversificationOfProtoFinnic} argues the uniform penetration of Slavic Christian loanwords into Balto-Finnic languages indicates the Balto-Finnic languages were still relatively close geographically and linguistically in the VIIIth century, an argument the author finds a strong indicator that the push of the Finns into Finland came with the pressure of the Slavic incursions into what would be the land of the Chuds, historically.
	Either way, there should be no Proto-Finns in Finland before the VIth century at the earliest, and they slowly displace the Saami in Finland as time goes on.
	
	The author think this narrative is more cohesive and matches available literature on barbaria better than the CK2 WtWSMS game history, which leans too heavily on Florin Curta’s work; his work is far from universally accepted and in CK2 WtWSMS, puts the Sklaveni and Antes near the Roman border in 476, where they were not even attested until the Early VIth Century, at least 50 years after the Western Part of the Empire fell. 
	
	\section{Timeline and Evolution}
	\label{sec:timeline}
	
	\subsection{The Situation in AD 350}
	\label{sec:timeline:subsec:350}
	
	Working from the previous Archelogical Complex map, we can begin to fill in a \textit{potential} culture map.
	So much of Barbarica remains unknown that besides cultures closely bordering Rome (Burgundians, Marcomanni, Quadi, Iazyges, Vandals, Taifals, Goths), we can only guess at how things are arranged.
	As such, this can be a point of contention and the author encourages others to make counterarguments.
	
	\begin{figure}[h!]
		\centering
		\includegraphics[width=1.00\textwidth]{./images/cultures_350AD.png}
		\caption{Culture proposal ca. AD 350.
			\newline\tiny The other bloc is a block of unknowns; probably a Germano-Dacian-Sarmatian mix, possibly Buri or Dacian Buris or even Victohali Germans}
	\end{figure}
	
	\begin{itemize}
		\item The Rygir are a giant unknown, but we postulate they simply migrated south from Pommerania; this is the most parsimonious path they could have taken, especially when the Marcomanni et al flee the Huns.
		\item The Skirians Heruli, and Gepids were all put adjoining the Goths.
		It is known that the former two raided the Black Sea (the Heruli with the Goths), and the Skirians would raid occasionally with the Dacians, so this seems to be a nice compromise.
		We put the Gepids at the source of the Dniester because it is mountainous (per \textit{Getica}'s reference to their lands) but puts them in a position to replaces the Vandals after the Vandals migrate.
		\item The Buri were listed by Tacitus as a Suebic Tribe but as a Lugii Tribe by Ptolemy.
		That mountainous location puts them just beyond the reach of the later Przeworsk \cite{HeatherEmpiresAndBarbarians,PrzeworskHistory}, but also next to the Marcomanni and Quadi, and would allow the Buri to be considering part of either group
		\item The Taifals are put in Oltenia where they settled in the late IIIrd Century.
		This puts them next to the Thervingi Goths (going by the East/West split ob the Dniester, which is a popular but not proven identification of the two), which their were tightly linked until the 370s.
		\item The Burgundians, of course, remain near the Rhine border, despite them being an East Germanic group.
		\item There is very little literature LT-Rascek can find for the Lebus, Gustow, and Denziner.
		More sources would be appreciated (2025-01-23)
	\end{itemize}
	
	To the east lie the Huns, who are about to assault the Alans and Goths...
	
	\section{Culture Review}
	\label{sec:culture_review}
	
	With the hard part over, we can then start working on redefining the various aspects of cultures. The author wanted to make certain the aesthetics were more consistent and fix some things he found annoying.
	
	\subsection{Aesthetics}
	\label{sec:culture_review:subsec:aesthetics}
	A few simple trigger overwrites allows for clothing groups to be more easily defined, instead of admixing them in a way that’s directly seen in the GUI. There’s no new clothing graphics, just new admixtures of existing graphics.
	
	I’ve also worked out how to make custom COA groups (more precisely, make a COA group comprised of other established groups without it throwing errors, so we can make more flexible COA configurations.
	
	Here are the new COA groups and clothing groups:
	
	\subsubsection{Fashion}
	\begin{itemize}
		\item East Germanic: Norse + Northern
		\item Sarmatian: Steppe + Northern
		\item Scythian: Turkic + Steppe
		\item Chernyakhov: Northern + Steppe
	\end{itemize}
	
	\subsubsection{Coat Of Arms}
	\begin{itemize}
		\item Gothic: Norse + Germanic + Baltic + East Slavic
		\item Vistulan: Germanic + Baltic + west Slavic
		\item Vandalic: Norse + Germanic + Baltic + West Slavic
		\item Chernyakhov: Germanic + East Slavic + Steppe
		\item Common Slavic: East Slavic + West Slavic + South Slavic
	\end{itemize}
	
	\subsubsection{Ethnicities}
	\begin{itemize}
		\item Vistulan: 50\% West Slavic, 25\% Germanic, 25\% Baltic
		\item Wielbark: Wielbark: 75\% Vistulan, 25\% Norse
		\item Gothic: 50\% Wielbark, 20\% Norse, 20\% East Slavic, 10\% Sarmatian
		\item Sciri: Unique Admixture of Asian, Sarmatian, Dacian, and Germanic
		\item Sarmatian: 50\% Vanilla Saka, 50\% Caucasian
		\item Sakan: 75\% Vanilla Saka, 25\% Caucasian
		\item Xiongnu: 75\% East Asian, 25\% Sakan
	\end{itemize}
	
	\subsection{Culture List}
	Updates from past version are \hl{highlighted}.
	
	\subsubsection{Common Slavic}
	
	Old Slavic:
	\begin{itemize}
		\item The core Slavic culture, sitting at the crossroads of East Germanic, Sarmatian, and Baltic civilization. Going with our idea of making them Chernyakhov, it occupies the forest expanse between the Wild Fields and Polesia.
		\newline
		Revamped Slavic Traditions to give them more flavor and to give them an easier time displacing tribal groups and migrating.
		\item Common Slavic Heritage, Common Slavic Language, Communal Ethos
		\item Forest Folk, \hl{East Slavic Traditions}, \hl{East Slavic Warfare}
	\end{itemize}
	
	\subsubsection{Baltic}
	
	\textbf{Dnieper Baltic:}
	
	Visutla Veneti:
	\begin{itemize}
		\item The Baltic Veneti we want to have, the same peoples as the core Dneiper Balts of Slavic history. I think because of the large number of unknowns, it should probably speak its own Baltic language rather than group it with East or West Baltic. It gets Wetlanders from its realm, Polygamous to influence later (likely) related Slav groups (Radmichi, \&c.), and Culture Blending to encourage conquerors removing it.
		\item Baltic Heritage, \hl{Venetic Baltic Language}, \hl{Stoic Ethos}
		\item \hl{Wetlanders}, \hl{Polygamous}, Culture Blending
		\item Continental Architecture, Northern Fashion, \hl{Vistulan COA}, \hl{Eastern Equipment}
		\item 50\% Baltic, 25\% East Slavic, 25\% Vistulan
	\end{itemize}
	
	East Galindian:
	\begin{itemize}
		\item The Galindians in the Russian Chronicles. Probably unrelated to the Galindians in the west but for having the same root for their autonym (probably meaning “at the extreme” in proto-Baltic).
		\item Baltic Heritage, \hl{Venetic Baltic Language}, Stoic Ethos
		\item \hl{Wetlanders}, \hl{Polygamous}, Culture Blending
		\item orest Wardens, Hill Dwellers, Staunch Traditionalists
		\item 50\% Baltic, 25\% East Slavic, 25\% Vistulan
	\end{itemize}
	
	\textbf{East Baltic:}
	
	\newpage
	
	\section{Bibliography}
	\label{sec:Bibliography}
	
	\begin{thebibliography}{10}
		\bibitem{HeatherEmpiresAndBarbarians}
		Heather, P. “Empire and Barbarians: The Fall of Rome and the Birth of Europe.”
		\bibitem{EmergenceOfRussia}
		Franklin, Simon; Shepard, Jonathan (2014-06-06). The Emergence of Russia 750–1200. Routledge. p. 101. ISBN 978-1-317-87224-5.
		\bibitem{BalticHydronyms}
		Baltic Hydronyms map.
		\bibitem{LaplandSaami}
		Akiko, Ante. "How did Lapland become Saami? Reconstructing the interaction of Proto-Saami, Proto-Norse and Palaeo-Laplandic language communities in the Iron Age". Presentation at "Contacts: Archaeology, genetics and languages"
		\bibitem{DiversificationOfProtoFinnic}
		Kallio, Petri. "The Diversification of Proto-Finnic" Fibula, Fabula, Fact: The Viking Age in Finland, pp. 155-168. Studia Fennica Historica 18. Helsinki. 2014.
		\bibitem{PrzeworskHistory}
		\textit{The Past Societies: Polish Lands from the First Evidence of Human Presence to the Early Middle Ages}. "Chapter 6: Przeworsk culture, society, and its long-distance contacts, AD 1-250". Institute of Archaeology and Ethnology, Polish Academy of Sciences.
		\bibitem{SaamiMap}
		Boradbent, N. D. "Lapps and Labyrinths:  Saami Prehistory, Colonization and Cultural Resilience." pg. 42. Smithsonian. 2010. \tiny Figure 29 provides a map of the probable Saami prehistorical settlement.
		\normalsize
		\bibitem{IndoEuroEncyclopedia}
		Mallory, J.P. and Adams, D.Q. "Encyclopedia of Indo-European Culture". Fitzroy Dearborn Publishers. 1997.
	\end{thebibliography}
	
	\newpage
	
	\section{Appendix I: New Ethnicities}
	\label{sec:appendix_new_ethnicities}
	
\end{document}