%For those not familiar with LaTeX, you can just put contributions in comments (preceded with % signs)
%so that LaTeX users can update the whole document
%LaTeX is preferred because, despite not being WYSIWYG, it is _very_ amenable to revision control, unlike
%MSOffice or LibreOffice software
%
%Try to keep one sentence per line for ease of maintenance
\documentclass{article}
\usepackage{fullpage}
\usepackage{graphicx}
\usepackage{geometry}
\usepackage{caption}
\usepackage{xcolor,soul}
\usepackage{comment}

\sethlcolor{yellow}

\graphicspath{./images}

\newcommand{\specificCite}[1]{\tiny #1 \normalsize}
\newcommand{\wtwsms}{\textit{WtWSMS}}

\title{Barbaria: Creating a Consistent Narrative for lands beyond the Roman border}

\begin{document}
	\newgeometry{left=1.5cm,bottom=1.5cm,right=1.5cm,top=1.5cm}

	\maketitle
	
	One issue with Germanic Paganism as a giant block is that it forces all Germanic peoples to have the exact same set of ritualism and tenets, which doesn't really reflect the variation that was seen within the whole Germanic population.
	Two very clear examples of things within CK3's engine that enforce a variation are inhumation vs. cremation and whether human sacrifice was a major component or not.
	
	Cremation remained typical in Norway into the Medieval period, and there's evidence of it throughout the Germanic period, although inhumation made inroads into the Przeworsk culture.
	By the Migration period, at least most of the Continental Germanics practiced inhumation, while it seems likely that the Angles practiced cremation, based on Anglo-Saxon practices (Angles mostly cremated and Saxons mostly inhumated).
	
	Human sacrifice \textit{is} mentioned by Tacitus and others, but it was not really strongly institutionalized until the Viking era, and Tacitus' writings only point to a small number of deaths.
	The polities where I can find laws outlawing human sacrifice are in continental Saxony, and some very clear references to human sacrifice such as the Blot with the Norse.
	
	So, based on those two practices, we have at least this split:
	
	\begin{tabular}{|c|c|c|}
		\hline
		Faith & Human Sacrifice & Burial Type\\
		\hline
		Norse & Yes & Cremation\\
		Saxon & Yes & Inhumation\\
		Prezworsk & No & Mostly Cremation, some Inhumation\\
		Continental & No & Mostly Inhumation\\
		\hline
	\end{tabular}
	
	At this point, it seems pretty clear practices are getting heavily split by heritage.
	Looking deeper into heritage, we can see additional splits into Continental up into Suebic/Irimonic, Frankish, and Gothic.
	Wolfram points out Goths had an emphasis on ancestor worship that wasn't as strong as with other Germanics (namely keeping tribe united sort of deal), the Franks were at least partially influenced by the Romans, and human sacrifice is oddly absent from Anglo-Saxon paganism. So based on that, we can probably split it like this
	
	\begin{tabular}{|c|c|c|}
		\hline
		Faith & Human Sacrifice & Burial Type\\
		\hline
		Norse & Yes & Cremation\\
		Saxon & Yes & Inhumation\\
		Prezworsk & No & Cremation\\
		Gothic & No & Inhumation\\
		Irimonic & No & Inhumation\\
		Frankish & No & Inhumation\\
		\hline
	\end{tabular}

	Additionally, we can remove Warmonger-like tendencies from the Irimonics, as the later Suebi seem to be a much more subdued people (they weren't utilized as federates or known for grand replacement-like conquests, unlike the Goths, Franks, Anglos, or Norse), so we can add that as another sort of split.
	
	\begin{tabular}{|c|c|c|c|}
		\hline
		Faith & Human Sacrifice & Burial Type & Warmonger? \\
		\hline
		Norse & Yes & Cremation & Yes\\
		Saxon & Yes & Inhumation & Yes\\
		Prezworsk & No & Cremation & No\\
		Gothic & No & Inhumation & Yes\\
		Irimonic & No & Inhumation & No\\
		Frankish & No & Inhumation & Yes\\
		\hline
	\end{tabular}
	
	This sort of split gives us a good idea on how to represent the Germanic Faiths
	
	\begin{thebibliography}{99}
	%Goths
	\bibitem{WolframHistoryOfTheGoths}
	Wolfram, H. \textit{History of the Goths}. University of California Press. 1990
	\bibitem{SaxonHumanSacrifice}
	<https://www.aldsidu.com/post/human-sacrifice-in-saxon-norse-historical-heathenry>
\end{thebibliography}
\end{document}